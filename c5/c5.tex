\documentclass{article}
\usepackage{../refalg}

\begin{document}
\Makepagesectionhead{MATH 593}{Tensor Product}{ARessegetes Stery}

\tableofcontents
\newpage

\section{Tensor Product of Modules}

\begin{definition}[$R$-balanced Maps]
    Let $R$ be a ring, with $M$ a right $R$-module, $N$ a left $R$-module and $P$ an abelian group. Then the map $\varphi: M \times N$ is \textbf{$\mathbf{R}$-balanced} if the followings are satisfied:
    \begin{itemize}
        \item $\varphi(u, v_1 + v_2) = \varphi(u, v_1) + \varphi(u, v_2)$ for all $u\in M, v_1, v_2 \in N$.
        \item $\varphi(u_1 + u_2, v) = \varphi(u_1, v) + \varphi(u_2, v)$ for all $u_1, u_2 \in M$, $v \in N$.
        \item $\varphi(ua, v) = \varphi(u, av)$ for all $a \in R, u\in M, b\in N$.
    \end{itemize}
\end{definition}

\begin{remark}
    The only difference between $R$-balanced maps and $R$-linear maps is the third condition: the coefficient in $R$ could be transferred between different positions, but not out of the expression.
\end{remark}

\begin{definition}[Tensor Product]
    A \textbf{tensor product} of $M$ and $N$ is an abelian group $M \tensor_R N$ with an $R$-balanced map $\varphi: M \times N \to M \tensor_R N$ which is universal w.r.t. the property: i.e. $\forall \psi: M \times N \to P$ which is $R$-balanced, there exists a unique $f: M \tensor_R N \to P$ s.t. $\psi = f \circ \varphi$ ($\psi$ factors uniquely through $\varphi$), i.e. the following diagram commute:
    \begin{figure}[htbp]
        \centering
        \begin{tikzcd}
            M \times N \arrow[rr, "\varphi"] \arrow[rrdd, "\psi"] & & M \tensor_R N \arrow[dd, dashed, "f"] \\
            & & \\
            & & P
        \end{tikzcd}
    \end{figure}
\end{definition}

\begin{remark}
    If $\tensor_R$ exists, then it is unique up to a canonical isomorphism. 
    
    Suppose that for $M, N \in \catlmod{R}$, there exists two tensor products $T$ and $T'$. Denote the canonical map from $M \times N$ to $T$ and $T'$ be $\varphi$ and $\varphi'$, respectively. Then by universal property of tensor product, there exists a unique isomorphism $f$ and $f'$ s.t. $f \circ \varphi = \varphi'$ and $f' \circ \varphi' = \varphi$, which gives $f \circ f' = \Id$.
\end{remark}

\begin{proposition}
    The tensor product exists.
\end{proposition}

\begin{proof}
    Proceed to show this via introducing relations on the free group structure. Let $F := \Z^{M \times N}$ be a free abelian group with basis $\{ e_{(u, v)} \mid (u, v) \in M \times N \}$. Quotient out the elements that are claimed to be equivalent by the constraint that the canonical map $\varphi$ should be $R$-balanced: consider $G \subseteq F$ to be generated by the following elements:
    \begin{itemize}
        \item $(e_{u_1 + u_2, v} - e_{u_1, v} - e_{u_2, v})$, for all $u_1, u_2 \in M, v \in N$.
        \item $(e_{u, v_1 + v_2} - e_{u, v_1} - e_{u, v_2})$, for all $u \in M, v_1, v_2 \in N$.
        \item $(e_{ua, v} - e_{u, av})$ for all $u \in M, v \in N, a \in R$.
    \end{itemize}
    By construction it is clear that the canonical map $\varphi: M \times N \to M \tensor_R N$ is $R$-balanced, via specifying $\varphi(u, v) = \overline{e_{u, v}}$. 

    It suffices to verify that the construction is compatible with the universal property. Consider the $R$-balanced map $\psi: M \times N \to P$, with the group homomorphism $g: F \to P$ s.t. $g(e_{u, v}) = \psi(u, v)$:
    \begin{figure}[htbp]
        \centering
        \begin{tikzcd}
            M \times N \arrow[rr, "\varphi"] \arrow[rrdd, "\psi"] & & F/G \arrow[dd, dashed, "f"] & & F \arrow[ll, "h"] \arrow[lldd, "g"] \\
            & & & & \\
            & & P & & \\
        \end{tikzcd}
    \end{figure}
    \begin{itemize}
        \item \emph{Existence}. Applying the universal property of quotient groups, which implies that there exists a unique $f$ s.t. $f \circ h = g$ where $h$ is the induced group homomorphism of the quotient. This is indeed valid, as $\psi$ is $R$-balanced, which by construction has kernel $G$.
        \item \emph{Uniqueness}. This follows from the result of universal property above; and the fact that $\varphi$ is surjective.
    \end{itemize}
\end{proof}

\begin{remark}\label{Rmk:decomp of tensor}
    The construction above, together with the fact that tensor products exist uniquely up to isomorphism, implies that for $R$-modules $M$ and $N$ with their system of generators, $(u_i)$ and $(v_i)$ respectively, for all $x \in M \tensor_R N$, there exists $(d_i) \in \Z$ s.t.
    \[
        x = \sum\limits_{i=1}^n d_i (u_i \tensor_R v_i)
    \]
    where the multiplication by integers is simply adding repetitively the elements to itself.
\end{remark}

The tensor products could also behave \emph{functorially}, via composing with the canonical map of tensor product:

Let $f: M \to M'$ a morphism of right $R$-modules, and $g: N \to N'$ a morphism of left $R$-modules. Then one could define a map $\psi: M \times N \to M' \tensor_R N'$, where $(u, v) \mapsto f(u) \tensor_R g(v)$. The map is $R$-balanced since the canonical map of tensor product is $R$-balanced. Therefore it is valid to apply the universal property of tensor product, which gives a unique group homomorphism $f: M \tensor_R N \to M' \tensor_R N' $. This is uniquely determined by $f$ and $g$; and is often denoted as $f \tensor_R g$. 

\begin{remark}
    This is also compatible with composition, via applying the universal property twice. Explicitly, for $f: M \to M', f': M' \to M''$ a morphism of right $R$-modules, and $g: N \to N', g': N' \to N''$ a morphism of left $R$-modules, we have
    \[
        (f' \tensor g') \circ (f \tensor g) = (f' \circ f) \tensor (g' \circ g)
    \]
\end{remark}

\begin{remark}
    In particular the constructions above induces a functor $M \tensor -: \catlmod{R} \to \catab$ for $M$ a right $R$-module, where 
    \[
        N \in \Ob(\catlmod{R}) \mapsto M \tensor N, \qquad f: N \to N' \mapsto \Id_{M} \tensor f
    \]
\end{remark}

Similar to the case of Hom Functors, we seek to lift the functor to $\catlmod{R} \to \catlmod{R}$. This requires extra structure on the module of interest. As before, making $R$ commutative, and restricting $M$ and $N$ to be both $R$-modules could resolve the issue, but the condition is too strong.  

\section{Bimodule}

\begin{definition}[Bimodule]
    Let $S$ and $R$ be rings. An \textbf{$\mathbf{R}$-$\mathbf{S}$ bimodule} $M$ is given by an an abelian group $M$ that is both a left $S$-module and a right $R$-module; and module operations is compatible, i.e.
    \[
        (au)b = a(ub) \qquad \forall u \in M, a \in S, b \in R
    \]
\end{definition}

\begin{remark}
    If $R$ is commutative, then every $R$-module is an $R$-$R$ bimodule (which is why making $R$ commutative suffices to ensure the Hom module has an $R$-module structure). In particular, $R$ is an $R$-$R$ bimodule.
\end{remark}

\begin{remark}
    Morphisms between $S$-$R$ bimodules inherits from corresponding modules. Compatibility does not interfere with morphisms.
\end{remark}

\begin{proposition}\label{Prop:Comp of bimodule}
    Let $R$ and $S$ be rings, with $M$ an $S$-$R$ bimodule, and $N$ a left $R$-module. Then there exists a unique left $S$-module structure on $M \tensor N$ s.t. $\lambda\cdot (u \tensor v) = (\lambda u) \tensor v$, for all $\lambda \in S, u\in M, v\in N$.
\end{proposition}

\begin{proof}
    Use the universal property, with $P = M \tensor N$. Fix $\lambda \in S$; consider $\varphi: M \times N \to M \tensor N$ s.t. $\varphi(u, v) = (\lambda u) \tensor v$. This map is $R$-balanced, as the tensor product on $R$ is balanced. 
    
    Then by universal property there exists a unique group homomorphism $f_{\lambda}: M \tensor N \to M \tensor N$, $u \tensor v \mapsto (\lambda u) \tensor v$. This gives the scalar multiplication of $\lambda$, which induces an $S$-module structure on $M \tensor N$.
\end{proof}

\begin{proposition}
    The extra structure on the modules gives extra structure on the morphisms in the universal property:

    Let $M$ be a $S$-$R$ bimodule, $N$ a left $R$-module, and $P$ a left $S$-module. Let $\varphi: M \times N \to M \tensor_R N$ the canonical map of tensor product. Suppose further that the map $\psi: M \times N \to P$ is $S$-bilinear. Then there exists a unique \underline{morphism of $S$-modules} $f: M \tensor_R N \to P$.
\end{proposition}

\begin{proof}
    By the universal property of tensor product, such morphism $f$ exists, and is uniquely specified by $f(u \tensor v) = \psi(u, v)$. It suffices to check that this is indeed a morphism of $S$-modules, i.e. for all $a \in S$, $f((au) \tensor v) = a f(u \tensor v)$. It then suffices to check that for certain (set of) fixed $u$ and $v$, as every element in $M \tensor N$ is of such form. This is clear as
    \[
        f((au) \tensor v) = \psi(au, v) \overset{!}{=} a\psi(u, v) = a\cdot f(u \tensor v)
    \]
    Equality (!) requires that $\psi$ is $S$-bilinear, and $M$ being a bimodule ensures that this is well-formed under the context of $S$-modules. 
\end{proof}

\begin{remark}
    The proposition above lifts the functor $M \tensor -$ to $\catlmod{R} \to \catlmod{S}$ for all $S$-$R$ bimodule $M$.
\end{remark}

\begin{remark}
    It may be interesting to consider the following property of bimodules:
    \begin{enumerate}
        \item If $R$ is commutative, then left or right $R$-modules are the same; and in this case $M \tensor_R N$ is an $R$-module. 
        \item If $M$ is a $T$-$R$ bimodule, and $N$ is an $R$-$S$ bimodule, then $M \tensor_R N$ is a $T$-$S$ bimodule.
    \end{enumerate}

    For the second remark, it is clear that $M \tensor_R N$ is both a left $T$-module, and a right $S$-module, via applying the same proof as in Proposition \ref{Prop:Comp of bimodule}. It suffices to prove that they are compatible. This is also clear from the construction in the proposition referred:
    \[
        (a(u \tensor_R v)) b = (au \tensor v) b = (au) \tensor (vb) = a(u \tensor (vb)) = a ((u\tensor v) b)
    \]
\end{remark}

\begin{remark}
    Let $R$ be a ring. Then $R$ is an $R$-$R$ bimodule. Let $M$ be a left $R$-module, which implies that $R \tensor_R M$ is a left $R$-module. Then there exists a functorial isomorphism $R \tensor_R M \simeq M$ for all $M \in \Ob(\catlmod{R})$. (This is called functorial as this could be regarded as the property of functor $R \tensor_R -$.)
\end{remark}

\begin{proof}
    Proof via using the universal property. Consider the morphism of $R$-modules $\alpha: R \times M \to M$, where $\alpha(a, u) = au$ for all $a \in R$, $u \in M$. It is $R$-linear, which is by definition $R$-balanced. The universal property gives that there exists a unique $f: R \tensor M \to M$ s.t. $f(a \tensor u) = au$. Designate $g: M \to R \tensor M$, $g(u) = 1 \tensor u$ for all $u\in M$. This is clearly $R$-balanced. This gives an isomorphism as $g \circ f = \Id_R$, $f \circ g = \Id_{R \tensor M}$.
\end{proof}

\section{Extension of Scalar}

Let $S$ and $R$ be rings, together with a ring homomorphism $\varphi: R \to S$. Then 
\begin{enumerate}
    \item It is clear that there is a \emph{restriction of scalar} functor:
    \[
        F: \catlmod{S} \to \catlmod{R}, \quad _SM \to {_RM} \qquad \text{ where } a\cdot u := \varphi(a)\cdot u\ \ (\forall a\in R, u\in {_SM})
    \]
    \item It is more interesting to consider the \emph{extension of scalar} functor $G: \catlmod{R} \to \catlmod{S}$. Notice that $\varphi$ gives $S$ a natural $R$-module structure, where $rs := \varphi(r)s$ for all $r \in R, s\in S$. This gives a natural extension of scalar functor $(S \tensor_R -)$:
    \begin{itemize}
        \item For $M \in \Ob(\catlmod{R})$, this gives $S \tensor_R M$.
        \item For $f: M_1 \to M_2$ a morphism of $R$-modules, this gives $\Id_S \tensor f$.
    \end{itemize}
\end{enumerate}

\begin{example}
    Consider the following examples:
    \begin{itemize}
        \item Let $\varphi: R \to R/I$ the canonical quotient map. Then this induces the isomorphism $G(M) \simeq M/IM$.
        
        Extension of scalar gives $G(M) \simeq R/I \tensor M$. To show that these two left $R/I$-modules are isomorphic, it suffices to specify maps between them s.t. the composition gives identity. Consider
        \[
            f: R/I \tensor M \to M/IM, \quad \bar{r} \tensor u \mapsto \overline{ru}, \qquad g: M/IM \to R/I \tensor M, \bar{u} \mapsto 1_{R/I} \tensor u
        \]
        It is clear that $f \circ g = \Id_{R/I \tensor M}$. Notice
        \[
            g \circ f (\bar{r} \tensor u) = g(\overline{ru}) = 1 \tensor \overline{ru} = 1 \tensor \bar{r} \cdot \bar{u} = \bar{r} \tensor \bar{u}
        \]
        since the canonical map of tensor product is $R$-balanced.
        \item Let $R$ be a ring, and $S \subseteq R$ a multiplicative system. Let $\varphi$ be the canonical map $R \to S^{-1}R$, $\varphi(a) = \frac{a}{1}$. Then this induces an isomorphism $G(M) \simeq S^{-1}M$.
        
        Apply the similar strategy. It suffices to show that $G(M) = S^{-1}R \tensor M \simeq S^{-1}M$. Consider
        \[
            f: S^{-1}R \tensor M \to S^{-1}M, \quad \frac{r}{s} \tensor u \mapsto \frac{ru}{s}, \qquad g: S^{-1}M \mapsto S^{-1}R \tensor M, \quad \frac{u}{s} \mapsto \frac{1}{s} \tensor u
        \]
        It is clear $f \circ g(\frac{u}{s}) = f(\frac{1}{s} \tensor u) = \frac{u}{s}$. For the other direction, check
        \[
            g \circ f (\frac{r}{s} \tensor u) = g(\frac{ru}{s}) = \frac{1}{s} \tensor (ru) = \left(\frac{1}{s}\right)\cdot r \tensor u = \frac{1}{s} \cdot \frac{r}{1} \tensor u = \frac{r}{s} \tensor u
        \]
        \item Tensor of module and localization of a ring is isomorphic to the localization of the module. Let $R$ be a commutative ring, $M$ be an $R$-module, and $U$ a multiplicative system in $R$. Then we have the isomorphism
        \[
            M_U \simeq M \tensor_R R_U
        \]
        Proof is done via constructing concrete maps. Consider the morphisms
        \[
            \begin{array}{c}
                f: M_U \to M \tensor_R R_U, \quad \dfrac{u}{s} \mapsto u \tensor \dfrac{1}{s} \\[8pt]
                g: M \tensor_R R_U \to M_U, \quad u \tensor \dfrac{r}{s} \mapsto \dfrac{ru}{s}
            \end{array}
        \]
        for all $u \in M, r \in R, s \in R\smallsetminus U$. First verify that these maps are indeed well-defined morphisms of $R$-modules:
        \begin{itemize}
            \item Consider $\frac{u_1}{s_1} \sim \frac{u_2}{s_2}$ where both of which are in $M_U$. By the definition of localization this indicates that there exists some $t \in R \smallsetminus U$ s.t. $t(s_1 u_2 - s_2 u_1) = 0$. This gives
            \[
                f\left( \frac{u_1}{s_1} \right) - f\left( \frac{u_2}{s_2} \right) = (s_2 u_1 - s_1 u_2) \tensor \frac{1}{s_1 s_2} = (t(s_2 u_1 - s_1 u_2)) \tensor \frac{1}{s_1 s_2 t} = 0
            \]
            which indicates that the image does not depend on the choice of representative.
            \item Consider $g': M \times R_U \to M_U, g'(u, \frac{r}{s}) = \frac{ru}{s}$. It is clear that $g'$ is bilinear as $R$ is commutative, which implies that $g'$ is $R$-balanced. Then $g$ exists by the universal property of tensor product.
        \end{itemize} 
        Then verify that the composition gives identity:
        \[
            f \circ g \left(u \tensor \frac{r}{s}\right) = ru \tensor \frac{1}{s} = u \cdot r \tensor \frac{1}{s} = u \tensor \frac{r}{s} \implies f \circ g = \Id_{M \tensor R_U}, \qquad
            g \circ f \left(\frac{u}{s}\right) = \frac{u}{s} \implies g \circ f = \Id_{M_U}
        \]
        This gives the desired isomorphism.
    \end{itemize}
\end{example}

\vspace*{1em}

\begin{theorem}[Universal Property of Extension of Scalars] 
    Let $M$ be a left $R$-module, $N$ be a left $S$-module, together with a ring homomorphism $\varphi: R \to S$. Let $f: M \to N$ a morphism of $R$-modules (in the sense of restriction of scalars). Then there exists a unique morphism of $S$-modules $g: S \tensor_R M \to N$ s.t. $g(1 \tensor u) = f(u)$ for all $u \in M$:
    
    \begin{figure}[htbp]
        \centering
        \begin{tikzcd}
            M \arrow[rr, "G"] \arrow[rrdd, "f"] & & S \tensor_R M \arrow[dd, dashed, "g"] \arrow[rr, <-, "\tensor"] & &   S \times M  \\
            & & \\
            & & N \arrow[rruu, <-, "\psi"]
        \end{tikzcd}
    \end{figure}
\end{theorem}

\begin{proof}
    Construct using the universal property of tensor product. Since $f$ is a morphism of $S$-modules, it is in particular $S$-linear, i.e.
        \[
            f(su) = s f(u) \implies \psi(s, u) = s f(u)
        \]
    There exists a canonical morphism into the setting of universal property, as $G(su) = \tensor(1, su) = \tensor(s, u)$ for all $s \in S, u \in M$; and $\psi$ is $R$-linear:
    \[
        \psi(sr, u) = (s\varphi(r)) f(u) = s (\varphi(r) f(u)) = s f(\varphi(r) u) = \psi(s, ru)
    \]
    \[
        \psi(ss', u) = ss' f(u) = s f(s'u) = \psi(s, s'u)
    \]
    Then the strengthened version of universal property gives the desired result.
\end{proof}

\begin{remark}
    Functorially, for tensor products we have the natural transformation (given the ring homomorphism $\varphi: R \to S$)
    \[
        \Hom_S(S \tensor_R M, N) \to \Hom_R(M, N), \qquad g \mapsto (u \mapsto g(1 \tensor u))
    \]
    The universal property gives that this is actually a bijection. Since $\Hom_R(M, N)$ is in essence the restriction of scalar, there is a bijection (which is exactly rephrasing the result above)
    \[
        \Hom_{S}(G(M), N) \simeq \Hom_R(M, F(N))
    \]
\end{remark}

\begin{definition}[Adjoint Pair]
    Consider functors $F: \catc \to \catd$, $G: \catd \to \catc$. They form an \textbf{adjoint pair} ($F, G$) if for all $a \in \Ob(\catc)$ and $b \in \Ob(\catd)$, we have a bijection $\Hom_{\catc}(G(b), a) \simeq \Hom_{\catd}(b, F(a))$ which is functorial w.r.t. both $a$ and $b$, i.e. there exists a natural transformation from $\Hom_{\catc}(G(-), a)$ to $\Hom_{\catd}(-, F(a))$.
\end{definition}

\begin{remark}
    Extension of scalar functor $G$ and restriction of scalar functor $F$ form an adjoint pair.
\end{remark}

\section{General Properties of Tensor Product}

\def\op{\text{op}}

\begin{proposition}
    Tensor product is \emph{commutative}, i.e. there exists an isomorphism of abelian groups $M \tensor_R N \simeq N \tensor_{R^{\op}} M$. Similarly, if $R$ is commutative, then this is an isomorphism of $R$-modules.
\end{proposition}

\begin{proof}
    Proceed via using the universal property of tensor product. Consider the map $\varphi: M \times N \to N \tensor_{R^{\op}} M$ given by $\varphi(v, u) = u \tensor_{R^{\op}} v$. It is clear $\varphi$ commutes with addition in either field. To show that $\varphi$ is indeed $R$-balanced it suffices to check the third property, which gives 
    \[
        \varphi(va, u) = u \tensor_{R^{\op}} (va) = u \tensor_{R^\op} a^\op v = u a^\op \tensor_{R^\op} v =(au) \tensor_{R^\op} v = \varphi(v, au)
    \]
    Similarly there exists an $R$-balanced map $\tilde{\psi}: N \times M \to M \tensor_R N$ (as $R^\op$ modules) which is given by $\psi(u, v) = v \tensor u$. This induces a map $\psi: N \tensor_{R^\op} M \to M \tensor_R N$. It is clear that $\varphi \circ \psi = \Id_{N \tensor_{R^\op} M}$, $\psi \circ \varphi = \Id_{M \tensor_R N}$. 
\end{proof}

\begin{remark}
    If $R$ is commutative, then the left $R$-modules are the same as right $R$-modules, which indicates that $M \tensor_R N \simeq N \tensor_R M$ (as in the commutative setting the opposite ring is the same as the original ring).
\end{remark}

\begin{proposition}
    Tensor product is \emph{associative}, i.e. for $M$ a right $R$-module, $N$ an $R$-$S$ bimodule, and $P$ a left $S$-module, there exists a unique isomorphism $f: (M \tensor_R N) \tensor_S P \to M \tensor_R (N \tensor_S P)$ s.t. $f((u \tensor_R v) \tensor_S w) = u \tensor_R (v \tensor_S w)$.
\end{proposition}

\begin{proof}
    Apply the universal property of tensor product twice. First consider map $f_z: M \times N \to M \tensor_R (N \tensor_S P)$ given by $f_z(x, y) = x \tensor_R (y \tensor_S z)$ for some $z \in P$. $f_z$ is $R$-balanced, as for all $a \in R$, 
    \[
        f_z(x, ay) = x\tensor_R (ay \tensor_S z) = x \tensor_R a(y \tensor_S z) = (xa) \tensor_R (y \tensor_S z) = f_z(xa, y)
    \]
    By universal property of tensor product this gives a unique map $\tilde{f_z}: M \tensor_R N \to M \tensor_R (N \tensor_S P)$, $f_z(x \tensor y) = x \tensor (y \tensor z)$. Now consider the map $f: M \tensor_R N \times P \to M \tensor_R (N \tensor_S P)$ given by $f(x \tensor y, z) = f_z(x, y)$. This is $S$-linear, as for all $a \in S$,
    \[
        f((x \tensor y)a, z) = f((x \tensor ya), z) = x \tensor (ya \tensor z) = x \tensor (y \tensor (az)) = f((x \tensor y), az)
    \]
    Similarly this gives a unique map $\tilde{f}: (M \tensor_R N) \tensor_S P \to M \tensor_R (N \tensor_S P)$. Repeat the process in the converse direction gives the inverse map, and it is clear that the composition of them is identity in the corresponding structure.
\end{proof}

\begin{proposition}
    Let $R$ be a commutative ring, and $f_1: R \to S_1$ and $f_2: R \to S_2$ be two $R$-algebras. Then there is a unique $R$-algebra structure on $S_1 \tensor_R S_2$ s.t. 
    \[
        (u_1 \tensor v_1) \cdot (u_2 \tensor v_2) = (u_1 u_2) \tensor (v_1 v_2)
    \]
\end{proposition}

\begin{proof}
    Since we are in the commutative setting, by using the associativity and commutativity of tensor product, there is an isomorphism $\Phi: (S_1 \tensor S_2) \tensor (S_1 \tensor S_2) \simeq (S_1 \tensor S_1) \tensor (S_2 \tensor S_2)$. By universal of property of $S_1 \tensor S_1$ and $S_2 \tensor S_2$, there exists a unique morphism of $R$-module $f_i(a_i \tensor b_i) = a_i b_i$ for all $a_i, b_i \in S_i$ with $i\in \{1, 2\}$. This gives $f_1 \tensor f_2: (S_1 \tensor S_1) \tensor (S_2 \tensor S_2) \to S_1 \tensor S_2$, which indicates that there is a unique map $f_1 \tensor f_2 \circ \Phi$ that maps $(S_1 \tensor S_2) \tensor (S_1 \tensor S_2) \to S_1 \tensor S_2$. Composing this with the tensoring of $(S_1 \tensor S_2)$ with itself gives the desired result.
\end{proof}

\begin{proposition}
    There exists an isomorphism of abelian groups $\Phi: \Hom_S(M \tensor_R N, P) \simeq \Hom_R(N, \Hom_S(M, P))$ for $N$ a left $R$-module, $P$ a left $S$-module, and $M$ an $S$-$R$ bimodule. 
\end{proposition}

\begin{proof}
    The only natural way to define this isomorphism is via $\Phi(\varphi) = (N \mapsto (M \mapsto \varphi(M \tensor N)))$, where for $f\in \Hom_S(M, P)$, $a \in R, u \in M$, $af(u) := f(ua)$. By construction this is a bijection, and additivity is satisfied in $P$.
\end{proof}

\begin{remark}
    This indicates that $M \tensor_R -$ and $\Hom_S(M, -)$ form an adjoint pair for $M$ being a $S$-$R$ bimodule. Furthermore, if $f: R \to S$ gives an $R$-algebra structure, then taking $M = S$ gives the adjoint pair of extension/restriction of scalar functors.
\end{remark}

\end{document}