\documentclass{article}
\usepackage{../refalg}

\begin{document}
\Makepagesectionhead{MATH 593 - Tensor Product}{ARessegetes Stery}

\tableofcontents
\newpage

\def\tensor{\otimes}

\section{Tensor Product of Modules}

\begin{definition}[$R$-balanced Maps]
    Let $R$ be a ring, with $M$ a right $R$-module, $N$ a left $R$-module and $P$ an abelian group. Then the map $\varphi: M \times N$ is \textbf{$\mathbf{R}$-balanced} if the followings are satisfied:
    \begin{itemize}
        \item $\varphi(u, v_1 + v_2) = \varphi(u, v_1) + \varphi(u, v_2)$ for all $u\in M, v_1, v_2 \in N$.
        \item $\varphi(u_1 + u_2, v) = \varphi(u_1, v) + \varphi(u_2, v)$ for all $u_1, u_2 \in M$, $v \in N$.
        \item $\varphi(ua, v) = \varphi(u, av)$ for all $a \in R, u\in M, b\in N$.
    \end{itemize}
\end{definition}

\begin{remark}
    The only difference between $R$-balanced maps and $R$-linear maps is the third condition: the coefficient in $R$ could be transferred between different positions, but not out of the expression.
\end{remark}

\begin{definition}[Tensor Product]
    A \textbf{tensor product} of $M$ and $N$ is an abelian group $M \tensor_R N$ with an $R$-balanced map $\varphi: M \times N \to M \tensor_R N$ which is universal w.r.t. the property: i.e. $\forall \psi: M \times N \to P$ which is $R$-balanced, there exists a unique $f: M \tensor_R N \to P$ s.t. $\psi = f \circ \varphi$ ($\psi$ factors uniquely through $\varphi$), i.e. the following diagram commute:
    \begin{figure}[htbp]
        \centering
        \begin{tikzcd}
            M \times N \arrow[rr, "\varphi"] \arrow[rrdd, "\psi"] & & M \tensor_R N \arrow[dd, dashed, "f"] \\
            & & \\
            & & P
        \end{tikzcd}
    \end{figure}
\end{definition}

\begin{remark}
    If $\tensor_R$ exists, then it is unique up to a canonical isomorphism. 
    
    Suppose that for $M, N \in \catlmod{R}$, there exists two tensor products $T$ and $T'$. Denote the canonical map from $M \times N$ to $T$ and $T'$ be $\varphi$ and $\varphi'$, respectively. Then by universal property of tensor product, there exists a unique isomorphism $f$ and $f'$ s.t. $f \circ \varphi = \varphi'$ and $f' \circ \varphi' = \varphi$, which gives $f \circ f' = \Id$.
\end{remark}

\begin{proposition}
    The tensor product exists.
\end{proposition}

\begin{proof}
    Proceed to show this via introducing relations on the free group structure. Let $F := \Z^{M \times N}$ be a free abelian group with basis $\{ e_{(u, v)} \mid (u, v) \in M \times N \}$. Quotient out the elements that are claimed to be equivalent by the constraint that the canonical map $\varphi$ should be $R$-balanced: consider $G \subseteq F$ to be generated by the following elements:
    \begin{itemize}
        \item $(e_{u_1 + u_2, v} - e_{u_1, v} - e_{u_2, v})$, for all $u_1, u_2 \in M, v \in N$.
        \item $(e_{u, v_1 + v_2} - e_{u, v_1} - e_{u, v_2})$, for all $u \in M, v_1, v_2 \in N$.
        \item $(e_{ua, v} - e_{u, av})$ for all $u \in M, v \in N, a \in R$.
    \end{itemize}
    By construction it is clear that the canonical map $\varphi: M \times N \to M \tensor_R N$ is $R$-balanced, via specifying $\varphi(u, v) = \overline{e_{u, v}}$. 

    It suffices to verify that the construction is compatible with the universal property. Consider the $R$-balanced map $\psi: M \times N \to P$, with the group homomorphism $g: F \to P$ s.t. $g(e_{u, v}) = \psi(u, v)$:
    \begin{figure}[htbp]
        \centering
        \begin{tikzcd}
            M \times N \arrow[rr, "\varphi"] \arrow[rrdd, "\psi"] & & F/G \arrow[dd, dashed, "f"] & & F \arrow[ll, "h"] \arrow[lldd, "g"] \\
            & & & & \\
            & & P & & \\
        \end{tikzcd}
    \end{figure}
    \begin{itemize}
        \item \emph{Existence}. Applying the universal property of quotient groups, which implies that there exists a unique $f$ s.t. $f \circ h = g$ where $h$ is the induced group homomorphism of the quotient. This is indeed valid, as $\psi$ is $R$-linear, which by construction has kernel $G$.
        \item \emph{Uniqueness}. This follows from the result of universal property above; and the fact that $\varphi$ is surjective.
    \end{itemize}
\end{proof}

\begin{remark}
    The construction above, together with the fact that tensor products exist uniquely up to isomorphism, implies that for $R$-modules $M$ and $N$ with their system of generators, $(u_i)$ and $(v_i)$ respectively, for all $x \in M \tensor_R N$, there exists $(d_i) \in \Z$ s.t.
    \[
        x = \sum\limits_{i=1}^n d_i (u_i \tensor_R v_i)
    \]
    where the multiplication by integers is simply adding repetitively the elements to itself.
\end{remark}

The tensor products could also behave \emph{functorially}, via composing with the canonical map of tensor product:

Let $f: M \to M'$ a morphism of right $R$-modules, and $g: N \to N'$ a morphism of left $R$-modules. Then one could define a map $\psi: M \times N \to M' \tensor_R N'$, where $(u, v) \mapsto f(u) \tensor_R g(v)$. The map is $R$-balanced since the canonical map of tensor product is $R$-balanced. Therefore it is valid to apply the universal property of tenbsor product, which gives a unique group homomorphism $f: M \tensor_R N \to M' \tensor_R N' $. This is uniquely determined by $f$ and $g$; and is often denoted as $f \tensor_R g$. 

\begin{remark}
    This is also compatible with composition, via applying the universal property twice. Explicitly, for $f: M \to M', f': M' \to M''$ a morphism of right $R$-modules, and $g: N \to N', g': N' \to N''$ a morphism of left $R$-modules, we have
    \[
        (f' \tensor g') \circ (f \tensor g) = (f' \circ f) \tensor (g' \circ g)
    \]
\end{remark}

\begin{remark}
    In particular the constructions above induces a functor $M \tensor -: \catlmod{R} \to \catab$ for $M$ a right $R$-module, where 
    \[
        N \in \Ob(\catlmod{R}) \mapsto M \tensor N, \qquad f: N \to N' \mapsto \Id_{M} \tensor f
    \]
\end{remark}

Similar to the case of Hom Functors, we seek to lift the functor to $\catlmod{R} \to \catlmod{R}$. This requires extra structure on the module of interest. Similarly, making $R$ commutative, and restricting $M$ and $N$ to be both $R$-modules could resolve the issue, but the condition is too strong.  

\section{Bimodule}

\section{Extension of Scalar}

\section{Adjoint Property of Tensor Product}

\end{document}