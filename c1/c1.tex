\documentclass{article}
\usepackage{../refalg}

\begin{document}
\Makepagesectionhead{MATH 593 - Ring}{ARessegetes Stery}

\tableofcontents
\newpage

\section{Ring homomorphism, Quotient Ring}

\begin{definition}[Ring Homomorphism]
    Let $X, Y$ be rings. A \textbf{Ring Homomorphism} is a map $f: X\to Y$ satisfying the following properties:
    \begin{itemize}
        \item $f(1) = 1$.
        \item $\forall x_1, x_2\in X, f(x_1) + f(x_2) = f(x_1 + x_2)$.
        \item $\forall x_1, x_2\in X, f(x_1 x_2) = f(x_1) f(x_2)$
    \end{itemize}
\end{definition}

\begin{definition}[Quotient Ring]
    Let $R$ be a ring and $I \subseteq R$ a two-sided ideal. The \textbf{Quotient Ring} $(R/I)$ is defined as $(R/\sim)$ with an equivalence relation $\sim$ where $a\sim b$ if and only if $a - b = I$. Elements in $(R/I)$ are denoted as $\bar{a}$, where $\bar{a} = \bar{b}$ if and only if $a\sim b$.
\end{definition}

The natural homomorphism $\pi_I: R\to (R/I)$ is defined as $\pi(a) = \bar{a}$, which satisfies the \emph{universal property of quotient rings}:

\begin{theorem}[Fundamental Theorem of Ring Homomorphisms]
    Let $\varphi: R\to S$ be a ring homomorphism, $I$ a two-sided ideal s.t. $I\subseteq \ker \varphi$, and $\pi$ be the natural ring homomorphism from $R$ to $(R/I)$. Then there exists a unique ring homomorphism $f: R/I \to S$ s.t. the following diagram commutes,
    \begin{figure}[htbp]
        \centering    
        \begin{tikzcd}[]
            R \arrow[rrdd, two heads, "\pi"] \arrow[rr, "\varphi"] & & S \\
            & & \\
            & & R/I \arrow[uu, "f"'] 
        \end{tikzcd}
    \end{figure}
    i.e. $\varphi = f\circ \pi$.
\end{theorem}

\begin{proof}
    It suffices to prove that $f$ exists and is unique, and verify that $f$ is indeed a ring homomorphism.
    \begin{itemize}
        \item \textbf{Uniqueness.} By the requirement that $f$ should make the diagram commute, $f(\bar{a}) = \varphi(a),\ \forall a\in R$. Uniqueness of $f$ follows from the fact that $\varphi$ maps every element in $R$ to a unique element in $S$.
        \item \textbf{Existence.} It suffices to verify that $f$ is well-defined, i.e. does not vary w.r.t. change of representative in $(R/I)$. For all $a, b\in R$ s.t. $\bar{a} = \bar{b}$, $(a - b)\in I \implies \varphi(a - b) = 0 \implies \varphi(a) = \varphi(b)$ since $\varphi$ is a ring homomorphism. By the uniqueness of $f$ it is specified that $f(\bar{a}) = \varphi(a)$, which implies that for all $\bar{a} = \bar{b}\in (R/I), f(\bar{a}) = \varphi(a) = \varphi(b) = f(\bar{b})$.
        \item \textbf{$f$ is indeed a homomorphism.} This follows from the fact that $\varphi$ is a ring homomorphism.
    \end{itemize}
\end{proof}

\newpage
\section{Ring of Fractions}

\begin{definition}[Multiplicative System]
    A subset $S\subseteq R$ for a ring $R$ is a \textbf{multiplicative system} if $1\in S$, and $\forall s_1, s_2\in S, s_1\cdot s_2\in S$, where $\cdot$ is the multiplication in $R$.
\end{definition}

\begin{definition}[Ring of Fractions]
    Let $R$ be a commutative ring, with $S\subseteq R$ a multiplicative subset, the \textbf{ring of fraction} $S^{-1}R$ is defined as $R\times S / \sim$, where $(s_1, r_1) \sim (s_2, r_2)$ if and only if there exists $t\in R$ s.t. $t(s_1 r_2 - s_2 r_1) = 0$. $(s, r) \in S^{-1}R$ is denoted as $\frac{s}{r}$. The definition of operations follows directly from analogy of that in $\Q$.

    The natural homomorphism (inclusion map) from $R$ to $S^{-1}R$ is defined as $r \hookrightarrow \frac{r}{1}$.
\end{definition}

\begin{remark}\upshape
    If $R$ is an integral domain, then $(s_1, r_1) \sim (s_2, r_2)$ iff $s_1 r_2 = s_2 r_1$, as for $\Q$.
\end{remark}

\begin{remark}
    If $R$ is not an integral domain, and $S$ contains zero divisors, then the inclusion map ceases to be injective, as choosing $t$ s.t. it satisfies $ts_1 = ts_2 = 0$ for some $s_1, s_2$ that are zero divisors gives $\varphi(s_1) = \varphi(s_2)$. Changing $R$ to an integral domain guarantees that the inclusion map $\varphi$ is injective.
\end{remark}

\begin{proposition}
    $\sim$ is an equivalence relation.
\end{proposition}

\begin{proof}
    It is clear that $\sim$ is reflexive and symmetric. For transitivity, consider $(s_1, r_1)\sim (s_2, r_2) \wedge (s_2, r_2)\sim (s_3, r_3)$. That is, there exists some $t_1, t_2\in R$ s.t. 
    \[
        \begin{cases}
            t_1 (s_1 r_2 - s_2 r_1) = 0 \\ t_2 (s_2 r_3 - s_3 r_2) = 0 
        \end{cases}
        \ \implies\ t_1 t_2 (s_1 r_2 s_3 - s_2 r_1 s_3) = t_1 t_2 (s_1 s_2 r_3 - s_2 r_1 s_3) = t_1 t_2 s_2 (s_1 r_3 - s_3 r_1) = 0
    \]
\end{proof}

\begin{remark}
    Notice that if $s\in S, then \frac{s}{a}$ for $a\in R$ is invertible. This tends more to a field, with more elements being ``reachable'' via multiplying an element from one side. A direct consequence is that less ideals exist in $S^{-1}R$, with ideals in $R$ whose generators differ by a factor that divides $s$ being identified in $S^{-1}R$. 
\end{remark}

\begin{remark}
    It is required that $R$ is commutative is to preserve the most structures from $R$, i.e. ensure that $S^{-1}I$ is an ideal for all ideals in $R$. This is due to the addition in action:
    \[
        \forall\ \frac{r_1}{s_1}, \frac{r_2}{s_2} \in S^{-1}R, \qquad \frac{r_1}{s_1} + \frac{r_2}{s_2} = \frac{r_1 s_2 + s_1 r_2}{s_1 s_2} 
    \]
    which indicates that $S^{-1}I$ is a two-sided ideal if and only if $I\subseteq R$ is a two-sided ideal. For one-sided (left/right) ideal the property is not fully inherited. 
\end{remark}

\begin{theorem}[Universal Property of Ring of Fractions]
    Suppose $R$ and $T$ are commutative rings, with $\varphi$ the inclusion of $R$ into $S^{-1}R$. Then for $f: R\to T$ s.t. $\forall s\in S, f(s)$ is invertible in $T$, there exists a unique ring homomorphism $g$ s.t. $f = g\circ \varphi$, i.e. make the following diagram commute:
    \begin{figure}[htbp]
        \centering    
        \begin{tikzcd}[]
            R \arrow[rrdd, "f"] \arrow[rr, hookrightarrow, "\varphi"] & & S^{-1}R \arrow[dd, "g"]  \\
            & & \\
            & & T
        \end{tikzcd}
    \end{figure}
\end{theorem}

\begin{proof}
    Adopt the same strategy as in the previous section: 
    \begin{itemize}
        \item \textbf{Existence.} For all $\frac{a}{s}\in S^{-1}R$, $g(\frac{a}{s}) := f(a) (f(s))^{-1}$ which is well-defined since $f$ is required to map all elements in $S$ to invertible elements. $g$ being a ring homomorphism follows from the fact that $f$ is a ring homomorphism. 
        \item \textbf{Uniqueness.} Follows from specifying $g(\frac{a}{s}) := f(a) (f(s))^{-1}$.
    \end{itemize}
\end{proof}

\begin{remark}
    If $S := R\smallsetminus\{0\}$, then $S^{-1}R$ is the whole field, with localization equivalent to completion of inverse of $R$.
\end{remark}

\subsection{Localization of a Ring}

\newpage
\section{Polynomial Rings}

\begin{definition}[R-algebra]
    Let $R$ be a ring. Then a ring $S$ is an \textbf{$R$-algebra} for the specific $R$ mentioned if there exists a ring homomorphism $\varphi: R\to S$ s.t. $\forall r\in R, s\in S, \varphi(r)s = s\varphi(r)$. When the homomorphism needs to be specified, the algebra is often denoted as a pair $\pair{S, \varphi}$
\end{definition}

\begin{remark}
    An $R$-algebra is a two-sided $R$-module, which can be regarded as a generalization of the structure in $R$. $R$ itself is not necessarily commutative, which implies that the associated homomorphism maps $R$ to the center of $S$.
\end{remark}

\begin{definition}[Morphism of $R$-algebras]
    Let $\pair{R_1, f_1}, \pair{R_2, f_2}$ be $R$-algebras. A \textbf{Morphism of $R$-algebras} is a ring homomorphism $\varphi: R_1 \to R_2$ s.t. the following diagram commute; i.e. $f_2 = \varphi \circ f_1$:
    \begin{figure}[htbp]
        \centering    
        \begin{tikzcd}[]
            R \arrow[rrdd, "f_2"] \arrow[rr, "f_1"] & & R_1 \arrow[dd, "\varphi"]  \\
            & & \\
            & & R_2
        \end{tikzcd}
    \end{figure}
\end{definition}

\begin{definition}[$R$-subalgebra]
    Let $\pair{S, f_s}$ be a $R$-algebra for $R$ a ring. $\pair{T, f_t}$ is a \textbf{$R$-subalgebra} of $S$ if $T$ is a $R$-algebra, with $f_t(R) \subseteq S$ ; and there exists a morphism $\varphi$ from $T$ to $S$, i.e. $\varphi$ makes the following diagram commute:
    \begin{figure}[htbp]
        \centering    
        \begin{tikzcd}[]
            R \arrow[rrdd, "f_s"] \arrow[rr, "f_t"] & & T \arrow[dd, hookrightarrow, "\varphi"]  \\
            & & \\
            & & S
        \end{tikzcd}
    \end{figure}
\end{definition}

\begin{definition}[Polynomial Ring]
    Let $R$ be a commutative ring. The \textbf{polynomial ring of $R$}, denoted $R[x]$, is defined as
    $$
        R[x] := \left\{ \sum\limits_{i=0}^{n} c_i x^i \mid n\in\N, c_i\in R \right\}
    $$
    with the addition and multiplication the same as in polynomials over $\Z$. The natural inclusion from $R$ to $R[x]$ is defined as $r\mapsto r$ which is a polynomial of degree 0.
\end{definition}

\begin{remark}
    If $R$ is a domain, then $R[x]$ is also a domain (consider the product of terms with highest degree); where $\deg (f g) \leq \deg (f) + \deg (g)$.
\end{remark}

\begin{theorem}[Universal Property of Polynomial Ring]
    Let $R$ be a ring and $\pair{S, f}$ an $R$-algebra, and $\varphi$ be the inclusion map from $R$ to $R[x]$. For all $s\in S$, there exists a unique morphism of $R$-algebra $g: R[x] \to S$ s.t. $g(x) = a$, and the following diagram commutes, i.e. $f = g\circ \varphi$:
    \begin{figure}[htbp]
        \centering    
        \begin{tikzcd}[]
            R \arrow[rrdd, hookrightarrow, "f"] \arrow[rr, "\varphi"] & & R[x] \arrow[dd, "g"]  \\
            & & \\
            & & S
        \end{tikzcd}
    \end{figure}
\end{theorem}

\begin{proof}
    Proceed similarly by first determining the form that $g$ takes, and then showing the uniqueness and existence. 
    \begin{itemize}
        \item \textbf{Uniqueness.} Since it is required that $g$ is a morphism of $R$-algebras, we have
        \[
            g\left( \sum\limits_{i=0}^n a_i x^i\right) = \sum\limits_{i=0}^{n} g(a_i) g(x^i) = \sum\limits_{i=0}^{n} f(a_i) g(x^i) = \sum\limits_{i=0}^{n} f(a_i) a^i
        \]
        by the requirement that $g(x) = a$. This is the only form that $g$ could take, and thus proves its uniqueness.
        \item \textbf{Existence.} For existence it suffices to check that $g$ is indeed a ring homomorphism. By the uniqueness $g$ is fixed by sending $x\in R[x]$ to $a\in R$. Notice that $R$ is commutative, which indicates that both left and right composition is satisfied; with the addition condition verified in the uniqueness part. 
    \end{itemize}
\end{proof}

\begin{theorem}[Universal Property of Polynomial Ring of Several Variables]
    Let $A$ be a commutative $R$-algebra and $g$ be the inclusion map from $R$ to $R[x_1, \cdots, x_n]$ with a fixed $n$. For every $R$-algebra $S$ and $(a_1, \cdots, a_n)\in S$, there exists a unique homomorphism of $R$-algebra $h: R[x_1, \cdots, x_n] \to S$ s.t. $h(x_i) = a_i$ for all $i\in \llbracket 1, n \rrbracket$, and the following diagram commutes, i.e. $f = h\circ g$:
    \begin{figure}[htbp]
        \centering
        \begin{tikzcd}[]
            R \arrow[rrdd, hookrightarrow, "g"] \arrow[rr, "f"] & & S \\
            & & \\
            & & R[x_1, \cdots, x_n] \arrow[uu, "h"]
        \end{tikzcd}
    \end{figure}
\end{theorem}

\begin{proof}[Sketch of Proof]
    The idea is similarly consider substitution $x_i \mapsto a_i$, and proceed to verify that this is indeed a ring homomorphism. One step that requires caution is that polynomials of several variables are defined in an inductive manner; therefore here proof should also be done inductively, on the number of variables involved. 
\end{proof}

Using polynomial of several variables, it is clearer to formalize the ``generating set'' of a ring via specifying which element each variable maps to:

\begin{definition}[Finitely Generated $R$-algebra]
    Let $R$ be a commutative ring, with $A$ a commutative $R$-algebra. Fix $(a_1, \cdots, a_n)\in A$. By the universal property of polynomial of several variables, there exists a unique homomorphism $\varphi: R[x_1, \cdots, x_n]$ s.t. $\varphi(x_i) = a_i$. Then the subalgebra $\im \varphi$ is said to be \textbf{generated} by $\{a_1, \cdots, a_n\}$. 
\end{definition}

\begin{remark}
    Using the same formalization as in the definition above, $\im \varphi$ is smallest $R$-subalgebra of $A$ that contains $\{ a_1, \cdots, a_n \}$. 
\end{remark}

\begin{proof}
    It is clear that $\im \varphi$ contains $\{ a_1, \cdots, a_n \}$. To see that it is smallest, suppose there is a smaller one $A'$, then there must be some $\sum_{i=0}^n a_i x^i \notin A'$, which contradicts with the fact that a ring should be closed. 
\end{proof}

Notice that in the definition of polynomial ring it is only required that $x$ could be multiplied with powers of itself. This enables making polynomial a representation of groups:

\begin{definition}[Group Ring]
    Let $R$ a commutative ring, and $G$ a group. A \textbf{group ring of $R$ on $G$} is defined as
    \[
        R[G] := \left\{ \sum\limits_{g\in G} a_g g \mid a_g \in R \right\}
    \]
    with the addition and multiplication the same as that in the polynomial ring. 
\end{definition}

\begin{remark}
    The operation between the ring and the group is not required to be defined and is simply a notation. The polynomial cannot admit any structure that is more complicated (e.g. changing the group to be a ring) as otherwise the addition will not be well-defined. 
\end{remark}

\section{Ideals}

\begin{definition}[Finitely-Generated Ideals]
    Let $R$ be a ring. Then
    \begin{itemize}
        \item Let $(I_{\alpha})$ be a family of ideals for $\alpha\in\Lambda$ the index set, then the \textbf{ideal generated by (sum of)} $(I_{\alpha})$ is defined as
        \[
            \sum\limits_{\alpha\in \Lambda' \subseteq \Lambda} I_{\alpha} := \left\{ \sum\limits_{\alpha\in \Lambda'} a_\alpha \Big| a_{\alpha} \in I_{\alpha}, \abs{\Lambda'} \text{ finite}  \right\}
        \]
        \item Alternatively one could consider the \textbf{ideal generated by (product of)} two ideals (which can be easily extended to several ideal cases) $I$ and $J$ to be
        \[
            I\cdot J := \left\{ \sum\limits_{i=1}^n a_i b_i \Big| n\in\Z_{>0}, a_i \in I, b_i\in J \forall i \right\}
        \]
        \item Suppose further that $R$ is commutative. Let $\Lambda := \{\lambda_1, \cdots, \lambda_n\}$ be a subset of $R$. Then the \textbf{ideal generated by} $\Lambda$ is defined as
        \[
            (\lambda_1, \cdots, \lambda_n) := \left\{ \sum\limits_{k=1}^n r_k \lambda_k \Big| r_k\in R \right\}
        \]
    \end{itemize}
\end{definition}

\vspace{1em}
\begin{remark}
    Ideals generated by only one element is principal. For finitely generated ideals, the ideal generated by a set of elements is the same as the ideal generated by the corresponding principal ideals of the elements. This simply results from the fact that $(a) = \left\{ ra | r\in R \right\}$.
\end{remark}

Specify $R$ to be a commutative ring, with $I \subseteq R$ an ideal of $R$. Consider the following special cases of ideals:

\begin{definition}[Radical Ideal]
    $I \subseteq R$ is a \textbf{radical ideal} if for all $a\in R$, $\exists n\in \Z_{>0}\ a^n\in I \implies a\in I$.
\end{definition}

\begin{definition}[Prime Ideal]
    $I \subseteq R$ is a \textbf{prime ideal} if $I \neq R$, and for all $a, b\in R, ab\in I \implies (a\in I) \vee (b\in I)$.
\end{definition}

\begin{definition}[Maximal Ideal]
    $I \subseteq R$ is a \textbf{maximal ideal} if $I \neq R$; and there is no ideal $J$ in $R$ s.t. $I \subsetneqq J \subsetneqq R$. 
\end{definition}

\begin{remark}
    Recall that $R$ is a domain if and only if for all $a, b\in R$, $ab = 0 \implies a = 0 \vee b = 0$. This implies that for any ring $R$ with $\mathfrak{p}$ a prime ideal in it, $R/\mathfrak{p}$ is a domain. 
\end{remark}

\begin{definition}[Reduced Ring]
    A $R$ is a \textbf{reduced ring} if and only if it does not have any nilpotent elements, i.e. for all $u\in R, u^n = 0 \implies u = 0$ for all $n\in \Z_{>0}$.
\end{definition}

\begin{remark}
    For a commutative ring $R$, $I$ is a radical ideal if and only if $R/I$ is a reduced ring.
\end{remark}

\begin{proposition}
    $I$ is a maximal ideal if and only if $R/I$ is a field. 
\end{proposition}

\begin{proof}
    This fact follows directly from the following simple lemma.
\end{proof}

\begin{lemma}
    $R = K$ is a field if and only if it only has two ideals $(0)$ and $(1)$.
\end{lemma}

\begin{proof}
    Consider in both directions:
    \begin{itemize}
        \item[$\Rightarrow$:] If $K$ is a field, then either there are no invertible elements, which in this case the ideal $I$ can only contain 0 as this is the only non-invertible element in a field; or $1$ and therefore every element is in the ideal, as $\forall g\in I, \exists g^{-1} \in K, gg^{-1} = 1 \in I$.
        \item[$\Leftarrow$:] If a ring $R$ has only two ideals $(0)$ and $(1)$, then for all $0\neq u\in R$ consider $(u)$. By hypothesis $(u) = (1)$, i.e. there exists some $u^{-1}\in R$, which implies that $R$ is actually a field.
    \end{itemize} 
\end{proof}

\begin{proposition}
    An ideal being maximal implies that it is prime; and an ideal being prime implies that it is radical. 
\end{proposition}

\begin{proof} 
   \emph{Maximal ideals are prime}. Suppose that $I \subseteq R$ is maximal but is not prime, i.e. there exists some $a, b\in R$ s.t. $ab\in R, a\notin R, b\notin R$. By hypothesis $I \cup \{a\} = R$., i.e. there exists some $r\in R, t\in I$ s.t. $a + rt = 1$. But then $b = ba + (br)t \in I$ which is a contradiction.

   \emph{Prime ideals are radical.} Consider inductively on $a$ and $a^{n-1}$; apply the definition of prime ideals.
\end{proof}

\begin{example}
    Consider counterexamples of the converse of the proposition above:
    \begin{itemize}
        \item $\Z_N$ for $N$ not a power of prime is radical, but not prime.
        \item A trivial case for an ideal being prime but not maximal is $(0)$, where as long as the ring is not a field, it is maximal.
        \item A more interesting case for an ideal being prime but not maximal is for finitely generated non-PIDs, adding a generator to a prime ideal suffices to create a ``larger'' ideal. Take the example $(x) \subseteq R[x]$ where $R$ is a domain, which is prime as $R[x]/\pair{x} \cong R$ is also a field. But $(x) \subseteq (2, x)$ which is not the whole ring.
    \end{itemize}
\end{example}

\section{Noetherian Ring}

\section{Euclidean Domain, PIDs and UFDs}

\end{document}