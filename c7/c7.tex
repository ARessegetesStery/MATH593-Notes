\documentclass{article}
\usepackage{../refalg}

\begin{document}
\Makepagesectionhead{MATH 593}{Multilinear Algebra}{ARessegetes Stery}

\def\catlalg#1{_{#1}\cat{Alg}}

\tableofcontents
\newpage

\section{The Tensor Algebra}

\begin{definition}[Multilinear]
    Let $R$ be a commutative ring, and $M_1, \cdots, M_n, N$ be $R$-modules. A map $\varphi: M_1 \times \cdots \times M_n \to N$ is \textbf{multilinear} if for all $i \in \llbracket 1, n \rrbracket$, for all $x_j \in M_j$ for $j \neq i$, the map $\varphi(x_1, \cdots, x_{i-1}, -, x_{i+1}, \cdots, x_n): M_i \to M$ is $R$-linear.
\end{definition}

\begin{remark}
    Via performing induction on $n$, it can be shown that for a multilinear map $f: M_1 \times \cdots \times M_n \to N$, with the tensor map $\varphi: M_1 \times \cdots \times M_n \to M_1 \tensor_R \cdots \tensor_R M_n$ (which is multilinear), there exists a $R$-linear map $g: M_1 \tensor_R \cdots \tensor_R M_n \to P$ s.t. $g \circ \varphi = f$.
\end{remark}

\begin{definition}[Tensor Algebra]
    Let $M$ be a fixed $R$-module. Define $T^0(M) := R, T^1(M) = M$; and for $n \geq 2$, define $T^n(M) := \underbrace{M \tensor_R \cdots \tensor_R M}_{n\ \text{times}}$. Then the \textbf{tensor algebra} is defined as
    \[
        T(M) := \bigoplus_{i \geq 0}T^i(M) = R \oplus M \oplus (M \tensor_R M) \oplus \cdots
    \]
\end{definition}

\begin{remark}
    Since for all $i$, $T^i(M)$ has an $R$-module structure, $T(M)$ is also an $R$-module.
\end{remark}

\begin{proposition}
    $T(M)$ also has an $R$-algebra structure.
\end{proposition}

\begin{proof}
    It suffices to define multiplication for each summand of $T(M)$ and check that it is well-defined. Define 
    \[
        \alpha_{ij}: T^i(M) \times T^j(M) \to T^{i+j}(M), \quad (a_1 \tensor \cdots \tensor a_i, b_1 \tensor \cdots \tensor b_j) \mapsto (a_1 \tensor \cdots \tensor a_i \tensor b_1 \tensor \cdots \tensor b_j)
    \]
    This is indeed well-defined, as by applying the universal property of tensor product for $i$ times gives the desired map. Notice that for the case where $i = 0$ or $j = 0$ this is just scalar multiplication, this is just scalar multiplication.
\end{proof}

\begin{remark}
    This can be extended to a map $T(M) \times T(M) \to T(M)$, which makes $T(M)$ a ring. The map is given by for $x = \oplus_{i \geq 0} x_i$, $y = \oplus_{j \geq 0} y_j$, the multiplication is defined as $x \cdot y = \oplus_{i, j \geq 0} \alpha_{i, j}(x_i, y_j)$, with $1 \in T^0(M) = R$. Moreover, the inclusion $R = T^0(M) \hookrightarrow T(M)$ is a ring homomorphism which makes $T(M)$ an $R$-algebra.
\end{remark}

\begin{remark}
    Notice that this differs from the polynomial ring in that it is not commutative (in terms of the direct summands). Therefore, the terms in $\oplus_{i, j} \alpha_{ij}$ cannot be collected into one. $T^n = M \tensor \cdots \tensor M$ has a basis given by $x_{i_1} \tensor \cdots \tensor x_{i_n}$, where $i_k \in \llbracket 1, d \rrbracket$ for all $k \in \llbracket 1, n \rrbracket$. Therefore, for $n \geq 2$, $T^n(M)$ is not commutative.
\end{remark}

\begin{proposition}[Universal Property of $T(M)$]
    Consider the (forgetful) functor $F: \catlalg{R} \to \catlmod{R}, M \mapsto T(M)$. Let $M$ be an $R$-module and $S$ an $R$-algebra. If $f: M \to S$ is a morphism of $R$-modules, then there exists a unique morphism of $R$-algebras $g: T(M) \to S$ s.t. $g\mid_{T^1(M)} = f$, i.e. $g \circ F = f$:
    \begin{figure}[htbp]
        \centering
        \begin{tikzcd}
            M \arrow[rr, hookrightarrow, "F"] \arrow[rrdd, "f"] & & T(M) \arrow[dd, dashed, "g"] \\
            & & \\
            & & S
        \end{tikzcd}
    \end{figure}
\end{proposition}

\begin{proof}
    Apply the universal property of tensor product and direct sum. 
    
    Define $f_n: \underbrace{M \times \cdots \times M}_{n\ \text{times}} \to S$, where $f_n(x_1, \cdots, x_n) = f(x_1)\cdots f(x_n)$. This is clearly multilinear, which implies that there exists a unique $R$-linear map $g_n: T^n(M) \to S$ s.t. $g_n(x_1 \tensor \cdots \tensor x_n) = f(x_1) \cdots f(x_n)$. Apply the universal property of direct sum on the superscript gives that there exists a unique $R$-linear map $g: T(M) \to S$ s.t. $g|_{T^n(M)} = g_n$. Check the followings:
    \begin{itemize}
        \item $g$ is a morphism of $R$-algebras. Since it is already a morphism of $R$-modules, it suffices to check that this definition is compatible with multiplication. For $x = \oplus_{i=0}^n x_i, y = \oplus_{j=0}^m y_j$, this gives
        \[
            g(x \cdot y) = g(x_0 \tensor \cdots \tensor x_n \tensor y_0 \tensor \cdots \tensor y_m) = \Pi_{i=0}^n f(x_i) \cdot \Pi_{j=0}^m f(y_j) = g(x_0 \tensor \cdots \tensor x_n) \cdot g(y_0 \tensor \cdots \tensor y_m) = g(x) \cdot g(y)
        \]
        \item $g$ is the unique morphism of $R$-algebras $T(M) \to S$, s.t. $g|_{T^1(M)} = f$. This is clear as defining $g|_{T^1(M)}$ gives the map on $g|_{T^n(M)}$ for all $n$, as by the definition of the multiplication. Furthermore, the map restricted to $T^0(M)$ is given by the associated morphism with the $R$-algebra $S$. Both of which are uniquely determined. 
    \end{itemize}
\end{proof}

\begin{remark}
    This makes $T$ a functor, which maps from $R$-modules to $R$-algebras. For all $R$-linear maps $f: M \to N$, there exists a unique morphism of $R$-algebras $T(f)$ s.t. the following diagram commutes:
    \begin{figure}[htbp]
        \centering
        \begin{tikzcd}
            M \arrow[dd, hookrightarrow] \arrow[rr, "f"] & & N \arrow[dd, hookrightarrow] \\
            & & \\
            T(M) \arrow[rr, "T(f)"] & & T(N)
        \end{tikzcd}
    \end{figure}
    This makes $T$ a functor as it preserves compositions. Further this is the left adjoint of the forgetful functor $G$ which only regards $S$ as an $R$-module, i.e. we have the following isomorphism
    \[
        \Hom_{\catlalg{R}}(T(M), S) \simeq \Hom_R(M, G(S))
    \]
    as by the universal property of tensor algebra the morphism from $T(M)$ to $S$ is uniquely defined by the map $f: M \to S$.
\end{remark}

\begin{definition}[Graded Ring]
    A ring $R$ is a \textbf{graded ring} if it comes with a decomposition $R = \oplus_{i \geq 0} R_i$ as abelian groups; and multiplication satisfies the relation $R_p \cdot R_q \subseteq R_{p + q}$ for all $p, q \geq 0$.
\end{definition}

\begin{remark}
    Consider the subring $R \subseteq R_0$. If $R_0$ lies in the center of $R$, i.e. $R_0 \subseteq \{a \in R \mid ab = ba \forall b \in R\}$, then $R$ becomes an $R_0$-algebra; and the decomposition $R = \oplus_{i \geq 0} R_i$ is a direct sum of $R_0$-modules.
\end{remark}

\begin{example}
    Consider the following examples of graded rings:
    \begin{enumerate}
        \item The tensor algebra $T(M)$ is a graded ring, where $R_0 = T^0(M) = R$
        \item The multivariate polynomials $S = R[x_1, \cdots, x_n]$ is a graded ring, where $S_d = \oplus_{\{i_1, \cdots, i_d \mid \sum_k i_k = d\}} R x_1^{i_1} \cdots x_d^{i_d}$.
    \end{enumerate}
\end{example}

\begin{definition}[Homogeneous]
    If $R = \oplus_{i \geq 0} R_i$ is a graded ring, the elements of $R_n$ are \textbf{homogeneous of degree $n$}.
\end{definition}

\begin{definition}[Morphism of Graded Rings]
    If $R$ and $S$ are graded rings, then a morphism of graded rings $f: R \to S$ is a ring homomorphism s.t. $f(R_i) \subseteq S_i$ for all $i$. Such definition gives the result that graded rings form a category.
\end{definition}

\begin{definition}[Homogeneous Ideal]
    If $R$ is a graded ring, and $I \subseteq R$ an ideal. $I$ is a \textbf{homogeneous ideal} if $I = \oplus_{i \geq 0} (I \cap R_i)$. Equivalently, for all $f \in I$, for all $f_i \in R_i$ s.t. $f = \sum_{i=0}^d f_i$, then $f_j \in I$ for all $j$. 
\end{definition}

\begin{remark}
    If further $I$ is two-sided, then $R/I = \oplus_{i \geq 0}(R_i / (R_i \cap I))$ as a direct sum of abelian groups. In this case, $R/I$ is a graded ring, and the quotient $\pi: R \to R/I$ is a morphism of graded rings.
\end{remark}

\begin{proposition}
    Let $R$ be a graded ring, and $I \subseteq R$ an ideal. Then $I$ is homogeneous if and only if it can be generated by homogeneous elements. 
\end{proposition}

\begin{proof}
    Show implication in two directions:
    \begin{enumerate}
        \item[$\Rightarrow$:] Since $I$ is homogeneous, there exists ideals $I_k \subseteq R_k$ s.t. $I = \oplus_{k \geq 0} I_k$. Then it is generated by the generating sets of $I_k$, which are all homogeneous.
        \item[$\Leftarrow$:] If $I$ can be generated by homogeneous elements, then for all $x \in R$ there exists a decomposition
        \[
            x = \sum_{r \in R} c_r r = \sum_{i \geq 0} \sum_{r \in R_i} c_{ri} r_i
        \]
        where only finitely many $c_{ri}$s can be non-zero, and $r_i \in I$. Collecting all the terms in the inner summation gives $x = \sum_{i \geq 0} c_i r_i$ for $r_i \in I_i \subseteq R_i$, which satisfies the definition of homogeneous ideals.
    \end{enumerate}
\end{proof}

\begin{remark}
    It is not necessary (and also not true) that all the homogeneous elements must (can) have the same degree. For example, it is completely valid to have $R \cdot (I \cap R_0) \subsetneq I \cap R_1$, which prevents any homogeneous generating set of the same degree from existing. 
\end{remark}

\section{Exterior and Symmetric Algebra}

\begin{definition}[Symmetric, Alternating Maps]
    Let $R$ be a commutative ring, with $M$ and $N$ $R$-modules. Let $\varphi: M^n \to N$ be a multilinear map. It is defined to be:
    \begin{enumerate}[label=\roman*)]
        \item \textbf{Symmetric}, if for all $\sigma \in S_n$, and for all $x_1, \cdots, x_n \in M$, $\varphi(x_1, \cdots, x_n) = \varphi(x_{\sigma(1)}, \cdots, \varphi(x_{\sigma(n)}))$.
        \item \textbf{Alternating}, if $\varphi(x_1, \cdots, x_n) = 0$ whenever $x_i = x_j$ for $i \neq j$. This is equivalent to stating that $\varphi$ is \textbf{skew-symmetric}, where $\varphi(x_1, \cdots, x_i, \cdots, x_j, \cdots, x_n) = -\varphi(x_1, \cdots, x_j, \cdots, x_i, \cdots, x_n)$ for all $i < j$.
    \end{enumerate}
\end{definition}

\begin{remark}
    Since all elements in $S_n$ (symmetric group) are generated by transpositions, to show that a map is symmetric it suffices to show the equality for all transpositions.
\end{remark}

\begin{definition}[Exterior Algebra]
    Let $T$ be the tensor algebra, with $J_n$ an $T^n$-submodule generated by $\{x_1 \tensor \cdots \tensor x_n \mid x_i = x_j\ \forall i \neq j\}$. Define $\Lambda^n(M) := T^n(M) / J_n$; and the \textbf{exterior algebra} $\Lambda(M) := \oplus_{n \geq 0} \Lambda^n(M)$. The algebra structure is inherited from that of the tensor algebra. 
\end{definition}

\begin{remark}
    Consider the ideal $J := \oplus_{i \geq 2} J_n \subseteq T(M)$. Here only summands with degree greater than or equal to 2 are taken into consideration as the definition ``alternating'' makes no sense for the lower degree cases. This is a two-sided ideal as tensor product is $R$-balanced; and each summand is an element in $J_n$ for some $n$. It is further homogeneous, as by definition. In this notation the exterior algebra can also be expressed as $\Lambda(M) := T(M)/J$.
\end{remark}

\begin{remark}
    The equivalence class of $x_1 \tensor \cdots \tensor x_n$ in $\Lambda(M)$ is often denoted as $x_1 \wedge \cdots \wedge x_n$, i.e. the \underline{wedge product}.
\end{remark}

\begin{proposition}[Universal Property of $\Lambda^n(M)$]
    The map $\varphi: M^n \to \lambda^n(M)$, $(x_1, \cdots, x_n) \mapsto x_1 \wedge \cdots \wedge x_n$ is an alternating multilinear map; and for all alternating multilinear map $\psi: M^n \to P$, there exists a unique $R$-linear map $f: \Lambda^n(M) \to P$ s.t. $f \circ \varphi = \psi$
\end{proposition}

\begin{proof}
    This follows directly from the definition of the exterior algebra, and the universal property of tensor product.
\end{proof}

\begin{example}
    If $g: M \to P$ is an $R$-linear map, then this gives an $R$-linear map $\Lambda^n g: \Lambda^n(M) \to \Lambda^n(P)$ for all $n$, s.t. $x_1 \wedge \cdots \wedge x_n \mapsto g(x_1) \wedge \cdots \wedge g(x_n)$. The construction implies that combining all $\Lambda^n g$s gives a morphism of graded $R$-algebra. 
\end{example}

\begin{proposition}
    If $(x_1, \cdots, x_d)$ is a system of generators of $M$, then $\Lambda^n(M)$ is generated as an $R$-module by $\{ x_{i_1} \wedge \cdots \wedge x_{i_n} \mid 1 \leq i_1 < \cdots < i_n \leq d \}$. In particular, for all $n > d$, $\Lambda^n(M) = 0$, which implies that $\Lambda(M)$ is a finitely generated $R$-module. 
\end{proposition}

\begin{proof}
    For the cases where $n \leq d$, the result follows from the fact that the tensor of several modules is generated by the tensor of the generators of the corresponding modules; and multilinear maps into the tensor product is alternating.
\end{proof}

\begin{proposition}
    $M$ is a free, finitely generated $R$-module, with basis $(x_1, \cdots, x_d)$. Then for all $n \leq d$, $\Lambda^n(M)$ is free with basis given by $\{x_{i_1} \wedge \cdots \wedge x_{i_n} \mid 1 \leq i_1 < \cdots < i_n \leq d\}$. Its rank is $\binom{d}{n}$.
\end{proposition}

\begin{proof}
    By the previous proposition, we only need to show that all the elements are linearly independent over $R$. Since the tensor product is commutative, fix the representation for $e_I$ for $I \subseteq \{1, \cdots, d\}$ to be $e_I = x_{i_1} \wedge \cdots \wedge x_{i_n}$ for $i_1 < \cdots < i_n$.

    Now consider the $I$s respectively. For a fixed $I$, define $\bar{I} := \{1, \cdots, d\} \backslash I$. By the fact that maps into the exterior algebra is alternating, we have
    \begin{equation}\label{eq:ext free}\tag{$\ast$}
        0 = e_{\bar{I}} \wedge \sum_{\abs{J} = n} a_J e_J = \sum_{\abs{J} = n} a_J (e_{\bar{I}} \wedge e_J) = a_I (x_1 \wedge \cdots \wedge x_n)
    \end{equation}
    We now seek to prove that $a_I = 0$ for all $I$, via apply a transformation into $R$. Consider the map $\psi: M^d \to R$ s.t. $\psi(u_1, \cdots, u_d) = \det (a_{ij})$, where $u_i = \sum_{i=1}^n a_{ij} x_j$, which is multilinear and alternating by construction. Then, by the universal property of exterior algebra, there exists a map $f: \Lambda^d(M) \to R$ s.t. $\psi(u_1, \cdots, u_d) = f(u_1 \wedge \cdots \wedge u_d)$. In particular, $f(x_1 \wedge \cdots \wedge x_d) = 1$. Applying $f$ to \eqref{eq:ext free} gives $a_I = 0$, which gives as a consequence the linear independence. 
\end{proof}

\begin{definition}[Symmetric Algebra]
    Define $I$ as the two-sided ideal generated by elements in the form of $\{x \tensor y - y \tensor x \mid x, y \in M\}$. The \textbf{symmetric algebra} $S(M) := T(M)/I$. This is a commutative $R$-algebra.
\end{definition}

\begin{remark}
    By construction $I$ is generated by homogeneous elements of degree 2, which is a homogeneous ideal (with $I_0 = I_1 = \{0\}$). This gives an alternative expression for the symmetric algebra 
    \[
        S(M) = \bigoplus_{n \geq 0} \frac{T^n(M)}{I \cap T^n(M)}
    \]
    which indicates that this is a graded ring. The denominator $I \cap T^n(M)$ is often denoted $S^n(M)$ or $\text{Sym}^n(M)$.
\end{remark}

\begin{proposition}[Universal Property of $S(M)$]
    $S(M)$ is a commutative $R$-algebra; and we have an inclusion $M \hookrightarrow S(M)$ which gives an isomorphism $M \simeq S^1(M)$. If $S$ is a commutative $R$-algebra, and $\beta: M \to S$ is a $R$-linear map, then there exists a unique $R$-algebra homomorphism $f: S(M) \to S$ s.t. $f \circ \alpha = \beta$.   
\end{proposition}

\begin{proof}
    By the universal property of $T(M)$, there exists a unique $R$-algebra homomorphism $\tilde{\beta}: T(M) \to S$ s.t. $\tilde{\beta} \mid_M = \alpha$. Since $S$ is commutative, $I \subseteq \ker (\tilde{\beta})$. By the universal property of quotient, there exists a unique morphism $\beta: S(M) \to S$ s.t. $\beta\mid_M = \alpha$.
\end{proof}

\begin{example}
    Let $M$ be a free $R$-module, with basis $x_1, \cdots, x_n$. The above universal property gives the isomorphisms where $S$ is a commutative $R$-algebra:
    \[
        \{ \text{$R$-algebra homomorphisms } S(M) \to S\} \simeq \{\text{$R$-linear maps $M \to S$}\} \simeq \{\text{maps } \{ x_1, \cdots, x_n \} \to S\}
    \]
    which implies that $S(M)$ satisfies the universal property of multivariate polynomials, i.e. we have the isomorphism $S(M) \simeq R[x_1, \cdots, x_n]$.
\end{example}

\begin{example}
    Let $\varphi: M^p \to T^p(M) \to S^p(M)$ be a symmetric multilinear map. For every symmetric multilinear map $\psi: M^p \to N$, there exists a unique $R$-linear map $f: S^p(M) \to N$ s.t. $\psi = f \circ \varphi$. This can be proved similarly using the universal property of quotient rings.
\end{example}

\section{Symmetric, Alternating and Hermitian Forms}

\section{The Spectral Theorem}

\end{document}