\documentclass{article}
\usepackage{../refalg}

\begin{document}
\Makepagesectionhead{MATH 593 - Module}{ARessegetes Stery}

\tableofcontents
\newpage

\section{Module}

\begin{definition}[$R$-Module]
    An (left) \textbf{$\bm{R}$-Module} $M$ is a set with two operations, often denoted as $(M, +, \times)$:
    \begin{itemize}
        \item Addition $(+): M\times M \to M$ , s.t. $(M, +)$ is an abelian group.
        \item Multiplication $(\times): R\times M \to M$, s.t. it has the following properties:
            \begin{itemize}
                \item \underline{Identity}. For all $x\in M$, there exists $1\in R$ s..t $1\cdot x = x$.
                \item \underline{Associativity}. For all $a, b\in R, x\in M$, $a(bx) = (ab)x$.
                \item \underline{Distributivity in $R$}. For all $a_1, a_2\in R$, $(a_1 + a_2)x = a_1 x + a_2 x$.
                \item \underline{Distributivity in $M$}. For all $a\in R, x_1, x_2\in M$, $a(x_1 + x_2) = a x_1 + a x_2$.
            \end{itemize}
    \end{itemize}
    Right modules are defined with the same structure, but with $a\times b = b\cdot a$ for $a\in R, b\in M$, where $\times$ is the multiplication in $M$, and $\cdot$ the multiplication in $R$.
\end{definition}

\begin{definition}[Submodule]
    Let $(M, +, \times)$ be an $R$-module. $N \subseteq M$ is a \textbf{$\bm{R}$-submodule} of $M$ if $(N, +)$ is a subgroup of $M$; and for all $n\in N, r\in R, n\times r\in N$.
\end{definition}

\begin{remark}
    Notice that $R$ itself gives an $R$-module, just as $\mathbb{K}$ gives a $\mathbb{K}$-vector space. Therefore $\pair{S, \varphi}$ an $R$-algebra induces a two-sided $R$-module structure. Check that this is indeed the case:
    \begin{itemize}
        \item \emph{Addition}. Adopt the addition in $S$ as a ring.
        \item \emph{Identity}: Since ring homomorphisms map identity to identity, $\varphi(1_R) = 1_S$, implying that $1_R$ is the identity for scalar multiplication.
        \item \emph{Associativity}. Results from the fact that multiplication in $S$ is associative.
        \item \emph{Distributivity in $R$/$M$}. Follows from the fact that $\varphi$ is a ring homomorphism.
    \end{itemize}
    In this sense, module generalizes the algebra structure. Generally one cannot ``revert'' the structure of a module back to an ideal. Specifically, suppose that $R$ is not commutative, then $R$ is not an $R$-algebra.
\end{remark}

\begin{remark}
    (Left) ideals of $R$ are submodules of $R$ taken as an $R$-submodule.
\end{remark}

\begin{remark}
    Let $M$ be an abelian group. Making $M$ into a (left) $R$-module is equivalent to specifying a ring homomorphism $\varphi: R\to \mathrm{End}(M)$, where $\mathrm{End}(\cdot)$ denotes the ring of endomorphisms on the specific structure. 
    
    It is worth noticing how the ring of endomorphism structure is defined. Specifically, the multiplication is the composition of endomorphisms on $M$. This can be viewed in two aspects:
    \begin{itemize}
        \item The associativity for $R$-modules is essentially stating that multiplication, i.e. elements of $R$ ``acting'' on those in $M$ is associative. Applying one action after another is the same as applying the composition of action.  
        \item Consider the definition of function as a set of pairs. Then
        \[
            R\times M \to M \cong (R\to M) \to M \cong R\to(M\to M)
        \]
        as the application of functions is associative. 
    \end{itemize}

    In particular, in the consideration of $\Z$-modules, the map $\varphi_{\Z}: \Z \to \mathrm{End}(M)$ is determined uniquely by the requirement that $1\mapsto 1_M = \Id_M$. Since addition and multiplication should be preserved, $n\mapsto n\cdot \Id_{M}$ for all $n\in \Z$. With the specification above one could observe the correspondence:
    \begin{itemize}
        \item $\{\Z \text{ modules}\} \Longleftrightarrow \{ \text{Abelian groups} \}$
        \item  $\{\Z/n\Z \text{ modules}\} \Longleftrightarrow \{ \text{Abelian groups } M \text{ s.t. } nx = 0\forall x\in M \}$
    \end{itemize}
\end{remark}

\section{Morphism of $R$-Modules}

\begin{definition}[Morphism of $R$-Modules]
    A \textbf{morphism of (left) $\bm{R}$-modules} $f: M\to N$ is an \underline{$R$-linear map}, which satisfies:
    \begin{itemize}
        \item $f(u_1 + u_2) = f(u_1) + f(u_2)$ for all $u_1, u_2\in M$.
        \item $f(au) = af(u)$, for all $u \in M, a\in R$.
    \end{itemize} 

    An \underline{isomorphism} of $R$-modules $f: M\to N$ is equivalently stating that
    \begin{itemize}
        \item There exists $g: N\to M$ s.t. $f\circ g = \Id_{M}$, $g\circ f = \Id_{N}$.
        \item $f$ is a bijection.
    \end{itemize}
\end{definition}

\begin{proposition}\label{prop:R-module morphism injective iff kernel is 0}
    Let $f: M\to N$ be a morphism of $R$-modules. Then $\im f\subseteq M$ and $\ker f \subseteq M$ are submodules; and $f$ is injective if and only if $\ker f = \{0\}$. 
\end{proposition}

\begin{proof}
    By the fact that $f$ is $R$-linear, both the image and kernel should be closed w.r.t. addition and scalar multiplication, i.e. are submodules. For the condition of injectivity, check
    \begin{itemize}
        \item[$\Rightarrow$:] Consider the contraposition. Suppose that $0 \neq a\in\ker f$. Then $f(1) = f(1 + a)$ with $1 \neq 1+a$ which is a contradiction.
        \item[$\Leftarrow$:] Consider the contraposition. Suppose that there exists $a \neq b \in R$ s.t. $f(a) = f(b)$, i.e. $f$ is not injective; then $f(a-b) = 0$ which indicates that $0 \neq (a - b)\in \ker $. 
    \end{itemize}
\end{proof}

\begin{definition}[Quotient Module]
    Let $N\subseteq M$ be a $R$-submodule. Define the equivalence relation $\sim: a\sim b$ if and only if $a - b \in N$. Then $M/N := M/\sim$ is a \textbf{quotient module}, with $\pi: m \to M/N$ the induced morphism of $R$-modules. 
\end{definition}

\begin{theorem}[Universal Property of Quotient Modules]
    \label{thm:Universal Property of Quotient Modules}
    Let $f: M\to P$ be a morphism of $R$-modules. Let $N$ be a submodule of $M$, with $\pi$ the induced morphism of $R$-modules. Further suppose that $N \subseteq \ker f$. Then there exists a unique $g: M/N \to P$ s.t. $f = g\circ \pi$, i.e. the following diagram commutes:
    \begin{figure}[htbp]
        \centering
        \begin{tikzcd}
            M \arrow[rr, two heads, "\pi"] \arrow[rrdd, "f"] & & M/N \arrow[dd, dashed, "g"] \\
            & & \\
            & & P \\
        \end{tikzcd}
    \end{figure}
\end{theorem}

\begin{proof}
    It suffices to verify that such map exists and is unique.
    \begin{itemize}
        \item \emph{Uniqueness}. Since the diagram is required to commute, if such function exists, it is fixed by $f(x) = g(\pi(x)) = g(\bar{x})$.
        \item \emph{Existence}. Then it suffices to check that $g$ such defined is indeed a morphism of $R$-modules. This is indeed the case as $f$ is a morphism of $R$-modules.
    \end{itemize}
\end{proof}

\begin{theorem}[First Isomorphism Theorem]\label{thm:First Isomorphism Theorem}
    Let $f: M\to N$ be a surjective morphism of $R$-modules. Define $K := \ker f$. If there exists a morphism of $R$-modules $\bar{f}: M/K\to N$ s.t. it is $R$-linear and $\bar{f}\circ \pi = f$, i.e. the following diagram commutes:
    \begin{figure}[htbp]
        \centering
        \begin{tikzcd}
            M \arrow[rr, "f"] \arrow[rrdd, "\pi"] & & N \\
            & & \\
            & & M/K \arrow[uu, dashed, "\bar{f}"]
        \end{tikzcd}
    \end{figure}

    Then $\bar{f}$ is an isomorphism.
\end{theorem}

\begin{proof}
    By the universal property of morphism of $R$-modules (Theorem \ref{thm:Universal Property of Quotient Modules}), a morphism $f: M/K\to N$ s.t. the diagram above commutes exists. It suffices to verify that $\bar{f}$ is bijective. It is surjective as $f$ is surjective; and is injective as $f(x) - f(y) = 0$ if and only if $(x - y)\in K$.
\end{proof}

\begin{definition}[Direct Product; Direct Sum]
    Let $(R_i)_{i\in I}$ be a family (potentially infinite) of modules. Then
    \begin{itemize}
        \item The \textbf{direct product} of them is the cartesian product $\prod_{i\in I} R_i$, where addition and multiplication is defined element-wise.
        \item The \textbf{direct sum} is a sub-ring of the direct sum $\bigoplus_{i\in I} R_i$ where only finitely many elements can be non-zero.
        \item $M$ is the \textbf{(internal) direct sum} if $M_1$ and $M_2$ if there exists an isomorphism $f: M_1 \oplus M_2 \to M$.
    \end{itemize}
\end{definition}

\begin{theorem}[Universal Property of Direct Product]\label{thm:Universal Property of Direct Product}
    Let $P$ be an $R$-module, $(M_i)_{i\in I}$ be a family of $R$-modules, with $f_j: P \to M_j$ a morphism of $R$-modules. Further let $p_j : \prod_{i\in I} M_i \to M_j$ the projection map s.t. $p_j(x) = x_j$ which is the $j$-th entry of the input. Then there exists a unique morphism of $R$-modules $f: P\to\prod_{i\in I} M_i$ s.t. $f(x) = (f_1(x), \cdots, f_n(x), \cdots)$; i.e. the following diagram commutes:
    \begin{figure}[htbp]
        \centering
        \begin{tikzcd}
            P \arrow[rr, dashed, "f"] \arrow[rrdd, "f_j"] & & \prod_{i\in I} M_i \arrow[dd, "p_j"] \\
            & & \\
            & & M_j
        \end{tikzcd}
    \end{figure}
\end{theorem}

\begin{proof}
    Uniqueness follows from the fact that $p_j\circ f$ should commute with $f_j$ for all $j$. Existence holds as $f_j$ is itself a morphism of $R$-modules.
\end{proof}

\begin{theorem}[Universal Property of Direct Sum]\label{thm:Universal Property of Direct Sum}
    Let $(M_i)_{i\in I}$ be a family of modules, with $f_j: M_j \to Q$ a family of morphism of $R$-algebras. Denote $\alpha_j$ to be the \underline{natrual embedding} s.t.
    \[
        \alpha_j: M_j  \to \bigoplus_{i\in I} M_i, \qquad \alpha_j(x) = (x_i)i, \quad\text{where } x_i = \begin{cases} x, & i = j \\ 0, & \text{otherwise} \end{cases}
    \]
    Then there exists a unique $R$-linear map $f: \bigoplus_{i\in I} M_i \to Q$ s.t. $f\circ \alpha_j = f_j$ for all $j$, i.e. the following diagram commutes:
    \begin{figure}[htbp]
        \centering
        \begin{tikzcd}
            M_j \arrow[rr, hookrightarrow, "\alpha_j"] \arrow[rrdd, "f_j"] & & \bigoplus_{i\in I} M_i \arrow[dd, dashed, "f"]\\
            & & \\
            & & Q
        \end{tikzcd}
    \end{figure}
\end{theorem}

\begin{proof}
    Since $f$ is required to be a morphism of $R$-modules, for all $x = (x_i)_{i\in I} \in \bigoplus_{i\in I} M_i$ it should satisfy the following conditions:
    \[
        f(x) = f\left( \sum\limits_{k\in I} \alpha_k(p_k(x)) \right) = \sum\limits_{k\in I} f(\alpha_k(p_k(x))) = \sum\limits_{k\in I} f_k(p_k(x))
    \]
    which is unique as $f_k$s and $p_k$s are uniquely defined. Since both $f_k$ and $p_k$ are homomorphisms, the composition is also a homomorphism.
\end{proof}

\section{Construction of Submodules}

This interlude provides some general constructions on how to obtain submodules of a given module. For the setup, let $R$ be a ring, with $M$ a left $R$-module.

\begin{enumerate}
    \item Let $(M_i)_{i\in I}$ be a family of submodules of $M$. Then $\bigcap_{i\in I} M_i$ is a submodule of $M$. 
    \item Consider the submodule generated by a subset $A \subseteq M$. By definition, $\pair{A} := \bigcup\{N \mid N\subseteq M, a\subseteq N, N \text{ submodules}\}$. The following proposition provides an explicit expression:
        \begin{proposition}\label{prop:Explicit Expression of Generated Submodules}
            The submodule generated by $A\subseteq M$ has the following explicit expression:
            \[
                \pair{A} = \left\{ \sum\limits_{i\in I} a_i x_i\ \Big|\ a_i \in R,\ x_i\in A, \text{ finitely many nonzero } a_i \right\}
            \]
        \end{proposition}

        \begin{proof}
            This is simply a re-formalization of the definition. Proceed by showing the double inclusion:
            \begin{itemize}
                \item[$\subseteq$:] Notice that RHS is indeed a module; and all elements in $A$ are contained in it by setting $a_i = 1$ and $x_i$ to be the desired element.
                \item[$\supseteq$:] By the fact that module should be closed w.r.t scalar multiplication and addition.
            \end{itemize}
        \end{proof}
    \item Let $(M_i)i\in I$ a family of modules. Then
        \[
            \sum\limits_{i\in I} M_i := \pair{\bigcup_{i\in I} M_i} := \left\{ \sum\limits_{i\in I} x_i \mid x_i\in M_i\ \forall i, \text{ finitely many nonzero } x_i \right\}
        \]
    \item It would be interesting to consider the following isomorphism of quotient of $R$-modules:
        \begin{theorem}[Third Isomorphism Theorem]\label{thm:Third Isomorphism Theorem}
            Let $M_1$ and $M_2$ be $R$-submodules of $M$. Then 
            \[
                (M_1 + M_2)/M_2 \cong M_1/(M_1 \cap M_2)  
            \]
         \end{theorem}

         \begin{proof}
            Consider two functions $f: M_1 \to (M_1 + M_2)/M_2$ and $g: M_1 \to M_1\cap M_2$. Attempt to show this via applying the first isomorphism theorem. Consider the following diagram:
            \begin{figure}
                \centering
                \begin{tikzcd}
                    M_1 \arrow[rr, two heads, "g"] \arrow[rrdd, "f"] & & M_1/(M_1\cap M_2) \arrow[dd, dashed, "h"] \\
                    & & \\
                    & & (M_1 + M_2)/M_2
                \end{tikzcd}
            \end{figure}

            In order to apply the first isomorphism theorem, it suffices to show that $M_1 \cap M_2 = \ker f$: as then the universal property grants the existence of such $h$, which allows the application of the First Isomorphism Theorem. This is indeed the case, as
            \begin{itemize}
                \item $M_1 \cap M_2 \subseteq \ker f$, as $M_1 \cap M_2 \subseteq M_2$ which is mapped to 0 by $f$.
                \item $M_1 \cap M_2 \supseteq \ker f$. For all $x\in \ker f$, by hypothesis $x\in M_1$; and the only elements that are annihilated by the quotient are those in $M_2$.
            \end{itemize}
        \end{proof}
    \item Let $N\subseteq M$ a left submodule. Let $I\subseteq R$ an ideal. Then consider the submodule
         \[
             IN := \left\{ IN := \sum\limits_{i\in\mathcal{I}} a_i x_i\ \Big|\ a_i \in I,\ x_i\in N,\ \text{finitely many nonzero } a_i \right\}
         \]
\end{enumerate}

\section{Free Modules}

\begin{definition}[Linear Combination (Module)]
    Let $M$ be an $R$-module, with $(x_i)_{i\in I}$ a finite family of elements in $M$. Then a \textbf{linear combination} of $x_i$s are for some fixed family of elements $(r_i)_{i\in I}$ in $R$ $\sum\limits_{i\in I} x_i r_i$.
\end{definition}

For the following definitions, fix $M$ to be an $R$-module.
\begin{definition}[System of Generators]
    $(x_i)_{i\in I} \subseteq M$ is a \textbf{system of generators} if $\pair{\{x_i \mid i\in I \}} = M$; i.e. every element in $M$ is a finite linear combination of generators.
\end{definition}

\begin{definition}[Finite Generation]
    $M$ is \textbf{finitely generated} if it admits a finite system of generators.
\end{definition}

\begin{definition}[Linear Independence]
    $A\subseteq M$ a subset of $M$ is \textbf{linearly independent} if the finite sum $\sum\limits_{a_i\in A, u_i\in U} a_i u_i = 0$ implies that for all $i$, $u_i = 0$.
\end{definition}

\begin{definition}[Basis]
    A basis of $M$ is an independent system of generators.
\end{definition}

\begin{definition}[Free Module]
    $M$ is a \textbf{Free $\bm{R}$-module} if it admits a basis.
\end{definition}

\begin{remark}
    $R$ not admitting a multiplicative inverse makes modules slightly different from vector spaces. Consider the following examples:
    \begin{enumerate}
        \item A nonzero module may not admit an independent subset. For example $R=\Z$ with $M = \Z/n\Z$. Then $n$ annihilates the whole ring.
        \item For $N\subseteq M$ a submodule, generally $M \cong N \oplus M/N$ does not hold. Take the example where $M = \Z$ and $N = n\Z$. $N \oplus (M/N)$ is not a domain as $n\cdot(0, 1) = (0, 0)$; but $M$ is effectively an integral domain.
        \item Similar to the case of vector spaces, it is useful to think in terms of modules in the canonical form. A useful result in vector space is that all $K$-vector spaces with dimension $n$ is isomorphic to $K^n$. We make the analogy in terms of modules. 
        
        Let $I$ be a set. Denote $R^{(I)} := \bigoplus_{i\in I} M_i = \left\{ (x_i)_{i\in I}\ \Big|\ \text{ finitely nonzero }x_i \right\}$, where $M_i = R$. This has a basis $(e_j)_{j\in I}$ which has 1 in the $j$-th entry. Every free (left) $R$-module is isomorphism to some $R^{(I)}$ which sends the bases to bases.
        \item If $R$ is commutative, then any two bases of a free $R$-module has the same cardinality (which is given by considering the quotient of maximal ideals and observe that every basis is a basis in the field; which has the same cardinality as this is in a vector space). But this can fail if $R$ is not commutative.
    \end{enumerate}
\end{remark}

\begin{theorem}[Universal Property of Free Modules]
    Let $F$ be a free $R$-module with basis $(e_i)_{i\in I}$, and $N$ an arbitrary $R$-module. For all $(u_i)_{i\in I} \subseteq N$, there exists a unique morphism of $R$-modules $f: F\to N$ s.t. $f(e_i) = u_i$ for all $i$.
\end{theorem}

\begin{proof}
    $f$ gives the definition and therefore restricts the map to be unique. The fact that both sides are bases ensures that this is a morphism of $R$-modules. 
\end{proof}

\begin{remark}
    The general thought is the same as where the universal property of ring homomorphisms of polynomial rings, where it is possible to decomposition the whole structure into several discrete structures; and designate maps on them correspondingly.
\end{remark}

\section{Finiteness Conditions on Modules}

\section{Modules of Finite Length}

\end{document}