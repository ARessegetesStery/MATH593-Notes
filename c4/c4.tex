\documentclass{article}
\usepackage{../refalg}

\begin{document}
\Makepagesectionhead{MATH 593 - Categories and Functors}{ARessegetes Stery}

\def\cat#1{\underline{\textup{#1}}}
\def\coker{\mathrm{coker}}
\def\catsets{\cat{Sets}}
\def\catrings{\cat{Rings}}
\def\catab{\cat{Ab}}
\def\catlmod#1{{}_{#1}\cat{Mod}}
\def\catrmod#1{\cat{Mod}_{#1}}
\def\cattop{\cat{Tops}}
\def\Ob{\mathrm{Ob}}

\def\catc{\mathcal{C}}
\def\catd{\mathcal{D}}

\tableofcontents
\newpage

\section{Category; Functor}

\begin{definition}[Category]
    A \textbf{category} $\catc$ consists of 
    \begin{itemize}
        \item A \underline{class} of objects $\catc$ (which, for example, could contain all sets), denoted $\Ob(\catc)$.
        \item For all $A, B \in \Ob(\catc)$, a \underline{set} $\Hom_{\catc}(A, B)$ the ``morphisms in $\catc$ from $A$ to $B$'' with map $\Hom_{\catc}(A, B) \times \Hom_{\catc}(B, C) \to \times \Hom_{\catc}(A, C)$ the ``morphism composition'' denoted $f \times g \rightsquigarrow (g\circ f)$, satisfying
        \begin{itemize}
            \item \emph{Existence of an identity}. for all $A \in \Ob(\catc)$, there exists $1_{A} \in \Hom_{\catc}(A, A)$ s.t. 
            \[
            \begin{cases}
                1_{A} \circ f = f & \forall f \in \Hom_{\catc}(A, B) \\
                g \circ 1_{A} = g & \forall g \in \Hom_{\catc}(B, A)
            \end{cases}
            \]
            \item \emph{Associativity}. For all $f \in \Hom_{\catc}(A, B), g\in \Hom_{\catc}(B, C), h \in \Hom_{\catc}(C, D)$, 
            \[
                (h \circ g) \circ f = h \circ (g \circ f)
            \]
        \end{itemize}
    \end{itemize}
\end{definition}

\begin{remark} The definition much resembles previous algebraic structures; but the morphisms and composition laws could be defined in a particularly strange way: 
    \begin{enumerate}
        \item Similar to monoids, the definition implies that the identity is unique. Suppose that there are two identities $1_{A}, 1_{A}' \in \Hom_{\catc}(A, A)$ for $A \in \Ob(\catc)$, then $1_A = 1_A \circ 1_A' = 1_A'$.
        \item The morphism is not necessarily a function; and in such cases composition needs to be re-defined respectively.
    \end{enumerate}
\end{remark}

\begin{example}
    Consider the following categories:
    \begin{itemize}
        \item \emph{Category of Sets} $\catsets$, where the objects are sets, and morphisms are maps between sets.
        \item \emph{Category of Rings} $\catrings$, where the objects are rings, and morphisms are ring homomorphisms.
        \item \emph{Category of (left) $R$-modules} $\catlmod{R}$, where objects are lef $R$-modules, and morphisms $R$-linear maps.
        \item Consider the category $\catc$ defined on a partially-ordered set $(A, \leq)$ where 
        \begin{itemize}
            \item $\Ob(\catc)$ consists of elements in $A$. 
            \item Morphisms are defined as
            \[
                \Hom_{\catc}(A, B) = \begin{cases}
                    \{\ast\} & A \leq B \\
                    \emptyset & \text{otherwise}
                \end{cases}
            \]
            where the composition of maps is defined as intersection. This is due to the fact that there can be no maps whose image is the empty set.
        \end{itemize}
    \end{itemize}
\end{example}

\begin{definition}[Functor]
    Let $\catc$ and $\catd$ be categories. The \textbf{functor} $F: \catc \to \catd$ consists of mappings for both objects and morphisms:
    \begin{itemize}
        \item For all $A \in \Ob(\catc)$, $F(A) \in \catd$.
        \item For all $f\in \Hom_{\catc}(A, B)$. $F(f) \in \Hom_{\catd} (F(A), F(b))$ s.t. 
        \begin{itemize}
            \item $F(1_A) = 1_{F(A)}$ for all $A \in \Ob(\catc)$. 
            \item For all $f \circ g$ where $f \in \Hom_{\catc}(A, B)$, $g \in \Hom_{\catc}(B, C)$, $F(f \circ g) = F(f) \circ F(g)$.
        \end{itemize}
    \end{itemize}
    The composition of functors is conducted in a natural way, i.e. applying consecutively. 
\end{definition}

\begin{example}
    Functors represent the induced maps w.r.t. a transformation in the structure:
    \begin{enumerate}
        \item Let $R$ be a commutative ring and $S \subseteq R$ a multiplicative system. Consider the functor $F: \catlmod{R} \to \catlmod{S^{-1}R}$ where
        \begin{itemize}
            \item $F(M) = S^{-1}M$ for all $M \in \Ob(\catlmod{R})$.
            \item For $f: M \to N$, define $F(f) := S^{-1}M \to S^{-1}N$, where $\frac{u}{s} \mapsto \frac{f(u)}{s}$.
        \end{itemize}
        \item Let $R$ be a ring, with $I \subseteq R$ a two-sided ideal of $R$; and $M$ a left $R$-module. Consider the functor $F: \catlmod{R} \to \catlmod{R/I}$ where
        \begin{itemize}
            \item $F(M) = M/IM$ for all $M \in \catlmod{R}$. 
            \item Let $f: M \to N$ be a morphism of left $R$-modules. Then it induces a map $\bar{f}: M/IM \to N/IN$ s.t. $\bar{f}(\bar(u)) = \overline{f(u)}$. Define $F(f) = \bar{f}$.
        \end{itemize}
        \item Functors generally can abandon structures. Let $M$ be a left $R$-module. By definition it is valid to view $M$ as an abelian group. Then functor $F: \catlmod{R} \to \catab$ where objects are taken to itself; and morphisms are taken to group homomorphisms. These are called \underline{forgetful functors}.
    \end{enumerate}
\end{example}

\section{Morphism of Categories}



\section{Products and Coproducts}

\section{Kernels and Cokernels}

For the motivation of kernel one could refer to \href{https://ncatlab.org/nlab/show/kernel}{\color{blue}{this}} article.

\section{Natural Transformations of Functors}
      
\end{document}