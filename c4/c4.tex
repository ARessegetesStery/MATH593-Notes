\documentclass{article}
\usepackage{../refalg}

\begin{document}
\Makepagesectionhead{MATH 593 - Categories and Functors}{ARessegetes Stery}

\def\cat#1{\underline{\textup{#1}}}
\def\coker{\mathrm{coker}}
\def\catsets{\cat{Sets}}
\def\catrings{\cat{Rings}}
\def\catfields{\cat{Fields}}
\def\catab{\cat{Ab}}
\def\catlmod#1{{}_{#1}\cat{Mod}}
\def\catrmod#1{\cat{Mod}_{#1}}
\def\cattop{\cat{Tops}}
\def\Ob{\mathrm{Ob}}
\def\dual#1{#1^{\circ}}

\def\catc{\mathscr{C}}
\def\catd{\mathscr{D}}

\tableofcontents
\newpage

\section{Category; Functor}

\begin{definition}[Category]
    A \textbf{category} $\catc$ consists of 
    \begin{itemize}
        \item A \underline{class} of objects $\catc$ (which, for example, could contain all sets), denoted $\Ob(\catc)$.
        \item For all $A, B \in \Ob(\catc)$, a \underline{set} $\Hom_{\catc}(A, B)$ the ``morphisms in $\catc$ from $A$ to $B$'' with map $\Hom_{\catc}(A, B) \times \Hom_{\catc}(B, C) \to \times \Hom_{\catc}(A, C)$ the ``morphism composition'' denoted $f \times g \rightsquigarrow (g\circ f)$, satisfying
        \begin{itemize}
            \item \emph{Existence of an identity}. for all $A \in \Ob(\catc)$, there exists $1_{A} \in \Hom_{\catc}(A, A)$ s.t. 
            \[
            \begin{cases}
                1_{A} \circ f = f & \forall f \in \Hom_{\catc}(A, B) \\
                g \circ 1_{A} = g & \forall g \in \Hom_{\catc}(B, A)
            \end{cases}
            \]
            \item \emph{Associativity}. For all $f \in \Hom_{\catc}(A, B), g\in \Hom_{\catc}(B, C), h \in \Hom_{\catc}(C, D)$, 
            \[
                (h \circ g) \circ f = h \circ (g \circ f)
            \]
        \end{itemize}
    \end{itemize}
\end{definition}

\begin{remark} The definition much resembles previous algebraic structures; but the morphisms and composition laws could be defined in a particularly strange way: 
    \begin{enumerate}
        \item Similar to monoids, the definition implies that the identity is unique. Suppose that there are two identities $1_{A}, 1_{A}' \in \Hom_{\catc}(A, A)$ for $A \in \Ob(\catc)$, then $1_A = 1_A \circ 1_A' = 1_A'$.
        \item The morphism is not necessarily a function; and in such cases composition needs to be re-defined respectively.
    \end{enumerate}
\end{remark}

\begin{example}
    Consider the following categories:
    \begin{itemize}
        \item \emph{Category of Sets} \catsets, where the objects are sets, and morphisms are maps between sets.
        \item \emph{Category of Rings} \catrings, where the objects are rings, and morphisms are ring homomorphisms.
        \item \emph{Category of (left) $R$-modules} $\catlmod{R}$, where objects are lef $R$-modules, and morphisms $R$-linear maps.
        \item Consider the category $\catc$ defined on a partially-ordered set $(A, \leq)$ where 
        \begin{itemize}
            \item $\Ob(\catc)$ consists of elements in $A$. 
            \item Morphisms are defined as
            \[
                \Hom_{\catc}(A, B) = \begin{cases}
                    \{\ast\} & A \leq B \\
                    \emptyset & \text{otherwise}
                \end{cases}
            \]
            where the composition of maps is defined as intersection. This is due to the fact that there can be no maps whose image is the empty set.
        \end{itemize}
    \end{itemize}
\end{example}

\begin{definition}[Functor]
    Let $\catc$ and $\catd$ be categories. The \textbf{functor} $F: \catc \to \catd$ consists of mappings for both objects and morphisms:
    \begin{itemize}
        \item For all $A \in \Ob(\catc)$, $F(A) \in \catd$.
        \item For all $f\in \Hom_{\catc}(A, B)$. $F(f) \in \Hom_{\catd} (F(A), F(b))$ s.t. 
        \begin{itemize}
            \item $F(1_A) = 1_{F(A)}$ for all $A \in \Ob(\catc)$. 
            \item For all $f \circ g$ where $f \in \Hom_{\catc}(A, B)$, $g \in \Hom_{\catc}(B, C)$, $F(f \circ g) = F(f) \circ F(g)$.
        \end{itemize}
    \end{itemize}
    The composition of functors is conducted in a natural way, i.e. applying consecutively. 
\end{definition}

\begin{example}
    Functors represent the induced maps w.r.t. a transformation in the structure:
    \begin{enumerate}
        \item Let $R$ be a commutative ring and $S \subseteq R$ a multiplicative system. Consider the functor $F: \catlmod{R} \to \catlmod{S^{-1}R}$ where
        \begin{itemize}
            \item $F(M) = S^{-1}M$ for all $M \in \Ob(\catlmod{R})$.
            \item For $f: M \to N$, define $F(f) := S^{-1}M \to S^{-1}N$, where $\frac{u}{s} \mapsto \frac{f(u)}{s}$.
        \end{itemize}
        \item Let $R$ be a ring, with $I \subseteq R$ a two-sided ideal of $R$; and $M$ a left $R$-module. Consider the functor $F: \catlmod{R} \to \catlmod{R/I}$ where
        \begin{itemize}
            \item $F(M) = M/IM$ for all $M \in \Ob(\catlmod{R})$. 
            \item Let $f: M \to N$ be a morphism of left $R$-modules. Then it induces a map $\bar{f}: M/IM \to N/IN$ s.t. $\bar{f}(\bar(u)) = \overline{f(u)}$. Define $F(f) = \bar{f}$.
        \end{itemize}
        \item Functors generally can abandon structures. Let $M$ be a left $R$-module. By definition it is valid to view $M$ as an abelian group. Then functor $F: \catlmod{R} \to \catab$ where objects are taken to itself; and morphisms are taken to group homomorphisms. These are called \emph{forgetful functors}.
    \end{enumerate}
\end{example}

\section{Morphism of Categories}

The dual of a category is where the direction of morphisms is inverted. The following gives a formalization of this:

\begin{definition}[Contravariant Functor]
    A \textbf{contravariant functor} $F: \catc \to \catd$ is a functor which maps composition to that in the inverse order, i.e.
    \begin{itemize}
        \item For all $A \in Ob(\catc)$, $F(A) \in \Ob(\catd)$.
        \item For all $f \in \Hom_{\catc}(A, B), F(f) \in \Hom_{\catc}(F(B), F(A))$ s.t. 
        \begin{itemize}
            \item $F(1_A) = 1_{F(A)}$ for all $A \in \Ob(\catc)$.
            \item For all $f \circ g$ where $f \in \Hom_{\catc}(A, B)$, $g \in \Hom_{\catc}(B, C)$, $F(f \circ g) = F(g) \circ F(f)$.
        \end{itemize}
    \end{itemize}
\end{definition}

\begin{definition}[Dual Category]
    Let $\catc$ be a category. Then the \textbf{dual category} $\dual{\catc}$ of $\catc$ is a category with
    \begin{itemize}
        \item $\Ob(\catc) = \Ob(\dual{\catc})$.
        \item $\Hom_{\catc}(A, B) = \Hom_{\dual{\catc}}(B, A)$. 
    \end{itemize}
    The composition is compatible as the inversion is done in the functor. 
\end{definition}

\begin{remark}
    Since the dual category is defined on the contravariant of the functor, replacing a functor with its contravariant is equivalent to replacing the category with its dual.
\end{remark}

Similar to the case of modules we can define the \emph{Hom Functors}; but as a concept one level up it leaves the image unspecified:

\begin{definition}[Hom Functor]
    Let $\catc$ be a category, and $A \in \Ob(\catc)$. Then the \textbf{Hom functor} $\Hom_{\catc}(A, -): \catc \to \catsets$ where
    \begin{itemize}
        \item For $B \in \Ob(\catc)$, $F(B) = \Hom_{\catc}(A, B)$.
        \item For $f: \Hom_{\catc}(B_1, B_2)$, $F(g): \Hom_{\catc}(A, B_1) \to \Hom_{\catc}(A, B_2)$, $g\mapsto f \circ g$. 
    \end{itemize}
\end{definition}

\begin{remark}
    Similarly, we can consider the contravariant functor of the Hom functor. $\Hom_{\dual{\catc}}(-, A) : \dual{\catc} \to \catsets$. By definition $\Hom_{\catc}(A, -) = \Hom_{\dual{\catc}}(-, A)$.
\end{remark}

\begin{remark}
    Let $\catc = \catlmod{R}$. Then $\Hom_{\catc}(X, -)$ could be lifted to $\catc \to \catab$. It can be further lifted to $\catc \to \catlmod{R}$ if $R$ is commutative, which ensures that the morphisms will be $R$-linear. In this case this is just the Hom Module of (left) $R$-modules. 
\end{remark}

\begin{definition}
    Let $\catc$ be a category. Then $u\in \Hom_{\catc}(A, B)$ is an \textbf{isomorphism} if there exists $v \in \Hom_{\catc}(B, A)$ s.t. $u\circ v = \Id_B, v\circ u = \Id_A$.
\end{definition}

\begin{remark}
    For a fixed $u$, such $v$ is unique. Suppose that there exists two distinct $v$s, we have
    \[
        v = v \circ \Id_B = v \circ (u \circ v') = (v \circ u) \circ v' = v'
    \]
    which is a contradiction.
\end{remark}

\begin{remark}
    Let $F: \catc \to \catd$ a functor. Then $u\in \Hom_{\catc} (A, B)$ being an isomorphism implies that $F(u)$ is an isomorphism.

    This results from the fact that $\Id_{F(B)} = F(\Id_B) = F(u \circ v) = F(u) \circ F(v)$. Result for $A$ is similar; and uniqueness follows from the same reasoning. 
\end{remark}

\begin{definition}
    Let $\catc$ be a category:
    \begin{itemize}
        \item $X \in \Ob(\catc)$ is an \textbf{initial object} if $\ \forall Y \in \Ob(\catc), \abs{\Hom_{\catc}(X, Y)} = 1$.
        \item $X \in \Ob(\catc)$ is a \textbf{final object} if $\ \forall Y \in \Ob(\catc), \abs{\Hom_{\catc}(Y, X)} = 1$.
        \item $X \in \Ob(\catc)$ is a \textbf{zero-object} if it is both an initial object and a final object. 
    \end{itemize} 
\end{definition}

\begin{remark}
    Let $X \in \Ob(\catc)$ be an initial (final, zero) object. Then $X'$ is initial (final, zero) if and only if there exists an isomorphism between $X$ and $X'$.

    Proof is similar for all three cases. Suppose that $X$ and $X'$ are both initial. Then there exists a unique $f \in \Hom_{\catc}(X, X')$ and $f'\in \Hom_{\catc}(X', X)$, i.e. $f' \circ f \in \Hom_{\catc}(X, X)$. $X$ being initial implies that this is the unique morphism from $X$ to itself, which contains $\Id_{X}$. Therefore $f' \circ f = \Id_{X}$. Similar result holds for $f \circ f' = \Id_{X'}$, which implies that $X$ and $X'$ are isomorphic.
\end{remark}

\begin{example}
    Although if initial/final objects are unique up to isomorphism if they exist, but they actually do not necessarily exist:
    \begin{enumerate}
        \item In $\catlmod{R}$, $\{0\}$ is a zero-object. 
        
        The only element in a left $R$-module that should be preserved in a morphism of $R$-module is the zero element. For any element $a \neq 0$, there exists two maps that either maps $a$ to 0, or another non-zero element, which indicates that this is not initial. 

        Suppose that the final object has (at least) two elements $\{0, a\}$ for $a \neq 0$, then there exists at least two maps from a non-trivial module generated by $(u_1, \cdots, u_r)$ to it: for each $u_i$ it is either mapped to $0$ or $a$, which gives two morphisms. 
        \item In \catsets, $\emptyset$ is initial, and $\{\ast\}$ (a set containing an arbitrary element) is final.
        
        Suppose that there exists an element in the initial object, then it could be mapped to any element as morphisms of \catsets do not have constraints. 
        \item In \catrings, $\Z$ is initial; and $\{0\}$ is final. 
        
        $\Z$ being initial results from the fact that ring homomorphisms are required to preserve the $0$ and $1$ elements; and the maximal ring generated by $(0, 1)$ is isomorphic to $\Z$.
        \item In \catfields, there are no initial or final objects.
        
        This results directly from the fact that $1 \neq 0$ in fields. For every two fields, there exists two maps: one that maps $1$ to $1$; and the other maps $1$ to $0$.
    \end{enumerate}
\end{example}

\begin{definition}
    Let $\catc$ be a category. $u\in \Hom_{\catc}(A, B)$ is a \textbf{monomorphism} if for all $C \in \Ob(\catc)$ and $v_1, v_2 \in \Hom_{\catc}(C, A)$ s.t. $u \circ v_1 = u \circ v_2$ implies $v_1 = v_2$. $u\in \Hom_{\catc}(A, B)$ is an \textbf{epimorphism} if for all $v_1, v_2$ with the same comdition above satisfies $v_1 \circ u = v_2 \circ u$ implies $v_1 = v_2$, i.e. it is a monomorphism in $\dual{\catc}$.
\end{definition}

\begin{remark}
    These are analogies of injective/surjective in the context of category. Since on the category level it is only valid to consider objects or morphisms, such analogies could be only made to morphisms.
\end{remark}

\begin{example}
    It is not always the case that monomorphisms could correspond to injective maps, and epimorphisms could correspond to surjective maps:
    \begin{enumerate}
        \item In \catsets, monomorphisms correspond to injective maps, and epimorphisms correspond to surjective maps.
        \item In $\catlmod{R}$, such analogy is still true via choosing the $v_1, v_2$:
        \begin{figure}[htbp]
            \centering
            \begin{minipage}{0.48\linewidth}
                \centering
                \begin{tikzcd}
                    \ker u \arrow[r, hookrightarrow, shift left, "\text{incl.}"] & \arrow[l, <-, shift left, "0"] A \arrow[r, "u"] & B
                \end{tikzcd}
            \end{minipage}
            \begin{minipage}{0.48\linewidth}
                \centering
                \begin{tikzcd}
                    A \arrow[r, "u"] & B \arrow[r, shift left, two heads, "\pi"] & \arrow[l, <-, shift left, "0"] B / \im u
                \end{tikzcd}
            \end{minipage}
        \end{figure}
        \begin{itemize}
            \item For $u$ being a monomorphism, $u(\ker u) = u(0) = 0$, i.e. the inclusion map from $\ker u$ to $A$ is the same as the zero map, i.e. $\ker u = \{0\}$. 
            \item For $u$ being an epimorphism, $\pi(u(A)) = 0$ in $B / \im u$, i.e. $\pi(B) = 0$ which indicates that $\im u = B$.
        \end{itemize}
        \item In \catrings, monomorphisms are still injective, via for $f: R\to S$ considering
        \begin{tikzcd}
            \Z[x] \arrow[r, "x \mapsto u", shift left] & \arrow[l, <-, shift left, "0"] R \arrow[r, "f"] & S
        \end{tikzcd}
        for all $u \in \ker f$. This implies that $u = 0$, i.e. $\ker f = \{0\}$.

        But epimorphisms in rings are not necessarily surjective. Take the example $\alpha: \Z \hookrightarrow \Q$ which is the inclusion map. This is an epimorphism as $v_1 \circ \alpha = v_2 \circ \alpha$ if and only if $1$ is mapped to the same element; but this always holds as ring homomorphisms preserve the multiplicative unit. 
    \end{enumerate}
\end{example}

\clearpage
\section{Products and Coproducts}

\begin{definition}[Product]
    Let $\catc$ be a category, with $(X_i)_{i\in I}$ a family of objects in $\catc$. Then the \textbf{product} of this family is given by an object $\Pi_{i\in I} X_i$ where for all $Y \in \Ob(\catc)$, with morphisms $q_j: Y \to X_j$ for $j\in I$, there exists a unique morphism $f$ s.t. $q_j = p_j \circ f$ for all $j$, i.e. the following diagram commute. The $p_j$ is the projection morphism, where $p_j (\Pi_{i\in I} x_i) = x_i$.
    \begin{figure}[htbp]
        \centering
        \begin{tikzcd}
            Y \arrow[rr, "f"] \arrow[rrdd, "q_j"] & & \Pi_{i\in I} X_i \arrow[dd, "p_j"] \\
            & & \\
            & & X_j
        \end{tikzcd}
    \end{figure}
\end{definition}

\begin{remark}
    Product of a family of objects is unique up to isomorphism. Suppose that there exists another product $X'$ which satisfies the criterion for being a product. Then there exists unique $\varphi$ and $\varphi'$ s.t. 
    \[
        \varphi: X \to X' \qquad \varphi': X' \to X
    \]
    since both $f$ and $f'$ are unique. But this gives 
    \[
        \begin{cases}
            p_j = p_j' \circ \varphi' \\
            p_j' = p_j \circ \varphi
        \end{cases}
        \implies
        \exists \varphi, \varphi' \text{ s.t. } \begin{cases}
            \varphi \circ \varphi' = \Id_{X'} \\
            \varphi' \circ \varphi = \Id_{X} \\
        \end{cases}
    \]
    By uniqueness this gives that $X$ and $X'$ must be isomorphic.
\end{remark}

\begin{definition}[Coproduct]
    Given a family $(X_i)_{i\in I} \in \Ob(\catc)$, the coproduct $\amalg_{i\in I} X_i$ is the product in the dual category, i.e. with all the arrow reversed. That is, for $Y \in \Ob(\catc)$ and $f_j: X_j \to Y$, denote $\alpha_j$ to be the natural embedding of $X_j$ into the product $\amalg_{i\in I} X_i$, then there exists a unique $f$ s.t. $f \circ \alpha_j = f_j$ for all $j$, i.e. the following diagram commute:
    \begin{figure}[htbp]
        \centering
        \begin{tikzcd}
            Y \arrow[rr, <-, "f"] \arrow[rrdd, <-, "f_j"] & & \Pi_{i\in I} X_i \arrow[dd, <-, "\alpha_j"] \\
            & & \\
            & & X_j
        \end{tikzcd}
    \end{figure}
\end{definition}

\begin{remark}
    In \catsets, the product is given by Cartesian product, and the coproduct is given by disjoint union. Notice the difference: projection has no corresponding morphism from disjoint union; and so is natural embedding into Cartesian product.

    In $\catlmod{R}$, the coproduct is given by the direct sum; and the product is given by the direct product. 
\end{remark}

\begin{remark}
    Product in the context of categories provides a generalization of Cartesian product, where specifying morphisms into the product gives the morphisms into each of its components. Coproduct, being the dual notion of product, simply ``reverses the arrows'', i.e. specifying morphisms from the coproduct gives morphisms from each of its components.
\end{remark}

\begin{definition}[Preadditive Category]
    A \textbf{preadditive category} $\catc$ is a category s.t. its morphisms form an abelian group, and is bilinear w.r.t. composition. 
\end{definition}

\begin{remark}
    \catrings\ is not a preadditive category, as its morphisms do not have a zero element (since ring homomorphisms are required to map $1$ to $1$.) $\catlmod{R}$, with out such constraint, is a preadditive category.
\end{remark}

\begin{definition}[Additive Functor]
    Let $\catc$ and $\catd$ be preadditive categories. $F: \catc \to \catd$ is an \textbf{additive functor} if for all $A, B \in \Ob(\catc)$, $F$ w.r.t. morphisms is a group homomorphism.
\end{definition}

\begin{remark}
    The definition of ``linear'', or ``group homomorphism'' implicitly requires that the underlying structure should have a valid operation. Those who don't, for example \catsets, are naturally excluded from such discussion.
\end{remark}

\begin{example}
    Let $R$ be a commutative ring and $I \subseteq R$ an ideal. Then the functor
    \[
        F: \catlmod{R} \to \catlmod{R}, \qquad M \mapsto M/IM
    \]
    is an additive functor. This results from the fact that morphism of $R$-modules $M \to N$ and the quotient morphism are both $R$-linear.
\end{example}

\begin{remark}
    Let $\catc$ be a preadditive category. Then
    \begin{enumerate}
        \item If $X \in \Ob(\catc)$ is an initial/final object, then it is a zero object. 
        
        Consider $\Hom_{\catc}(X, X)$. Since $\catc$ is preadditive, this forms a group of one element, which is zero. Therefore for all $f \in \Hom_{\catc}(X, X)$ this gives $f = 1_X \circ f = 0 \circ f = 0$. This immediately implies that $X$ is a zero object, as for all $g \in \Hom_{\catc}(Y, X)$, $h \in \Hom_{\catc}(X, Y)$
        \[
            g = g \circ 1_{X} = g\circ 0 = 0, \qquad h = 1_X \circ h = 0 \circ h = 0
        \]
        \item $\catc$ being preadditive implies that $\dual{\catc}$ is preadditive, as reversing the arrow does not interfere with the group structure, or the additive property.
        \item Let $F: \catc \to \catd$ an additive functor, with $\catc$ and $
        \catd$ preadditive categories. Then for $0 \in \Ob(\catc)$ the zero object in $\catc$, $F(0)$ is also the zero object in $\catd$.
        
        This results directly from the fact that group homomorphisms map $0$ to $0$; and by the first point in the remark, an object $X$ is zero if and only if $1_X = 0$ in $\Hom_{\catc}(X, X)$.
    \end{enumerate}
\end{remark}

\section{Kernels and Cokernels}

For the motivation of kernel one could refer to \href{https://ncatlab.org/nlab/show/kernel}{\color{blue}{this}} article.

\section{Natural Transformations of Functors}
      
\end{document}