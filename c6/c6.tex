\documentclass{article}
\usepackage{../refalg}

\begin{document}
\Makepagesectionhead{MATH 593}{Introduction to Homological Algebra}{ARessegetes Stery}

\tableofcontents
\newpage

\def\too{\longrightarrow}

\section{Exactness}

\begin{definition}[Complex]
    A \textbf{Complex} of $R$-modules is a family of $R$-modules $(M_i)$ and $R$-linear maps $d_i: M_i \to M_{i+1}$ s.t. for all $i$, $d_{i+1} \circ d_i = 0$.
\end{definition}

\begin{remark}
    The followings are some specifications on the notations:
    \begin{itemize}
        \item The complex is often denoted by a chain
        \[
            \cdots \overset{d_{i-2}}{\too} M^{i-1} \overset{d_{i-1}}{\too} M^i \overset{d_{i}}{\too} M^{i+1} \overset{d_{i+1}}{\too} \cdots
        \]
        or a chain with indices on the bottom with $M^i = M_{-i}$.
        \item The complex extends to infinity in both ends. If the notation terminated on one side, all modules not written out are the trivial (the zero module).
    \end{itemize}
\end{remark}

\begin{remark}
    The definition of a complex is the same as stating that $\im d_{i} \subseteq \ker d_{i+1}$ for all $i$.
\end{remark}

\begin{definition}
    For a sequence $A \overset{f}{\too} B \overset{g}{\too} C$ where $f$ and $g$ are $R$-linear maps, it is \textbf{exact at $\mathbf{B}$} if the equality is reached in the remark above, i.e. $\im f = \ker g$. 

    A sequence is exact if it is exact at $A_i$ for all $i$. A complex is exact if it is exact everywhere.
\end{definition}

\begin{example}
    The sequence $0 \too A \overset{f}{\too} B$ is exact implies that $\ker f = \{0\}$, i.e. $f$ is injective. Similarly, $A \overset{g}{\too} B \too 0$ implies that $g$ is surjective.
\end{example}

\begin{definition}
    A \textbf{Short Exact Sequence (SES)} is an exact sequence
    \[
        0 \too M' \overset{i}{\too} M \overset{p}{\too} M'' \too 0
    \]
\end{definition}

\begin{proposition}
    Given a sequence $(\ast): 0 \too M' \overset{i}{\too} M \overset{p}{\too} M'' \too 0$, the followings are equvalent:
    \begin{enumerate}[label=\roman*)]
        \item $(\ast)$ is a short exact sequence.
        \item $i$ is injective, and for $q: M \to \coker i$, there exists a unique isomorphism $\mu$ s.t. $\mu \circ q = p$.
        \item $p$ is surjective, and for $j: M \to \ker p$, there exists a unique isomorphism $\eta$ s.t. $i = \eta \circ j$.
    \end{enumerate}
\end{proposition}

\begin{proof}
    It suffices to prove the equivalence between i) and ii), as the case with iii) is similar:
    \begin{itemize}
        \item \emph{i) $\Rightarrow$ ii)}. Apply the universal property of cokernel. Since $(\ast)$ is exact, $p \circ i = 0$, there exists a map $\theta$ s.t. the following diagram commutes.
        \begin{figure}[htbp]
            \centering
            \begin{tikzcd}
                M' \arrow[rr, "i"] & & M \arrow[rr, "p"] \arrow[rrdd, "q"] & & M'' \arrow[dd, <-, "\theta"] \\
                & & & & \\
                & & & & \coker i
            \end{tikzcd}
        \end{figure}
        The fact that $p$ is surjective, and the diagram should commute gives $\theta$ should be surjective. To prove that $\theta$ is injective, it suffices to verify that $\theta(b) = 0 \implies b = 0$ for $b \in \coker i$. Since $q$ by definition is surjective, there exists $a \in M$ s.t. $q(a) = b$. This gives $a \in \ker p = \im i$, which implies that $q(a) = 0$ as the cokernel is defined by $M/ \im i$.
        \item \emph{ii) $\Rightarrow$ i)}. Given that $\mu$ is an isomorphism and $i$ is injective, it suffices to verify that $p$ is surjective, and $\im i = \ker p$. $\mu$ being surjective implies that $p$ is surjective; and $\mu$ being an isomorphism implies that $\ker p = \ker q = \im i$. 
    \end{itemize}
\end{proof}

\begin{proposition}
    Given a short exact sequence $0 \too M' \overset{i}{\too} M \overset{p}{\too} M'' \too 0$, the following statements are equivalent:
    \begin{enumerate}[label=\roman*)]
        \item There exists $q: M \to M'$ s.t. $j \circ i = \Id_{M'}$
        \item There exists $j: M \to M''$ s.t. $q \circ p = \Id_{M''}$
        \item There exists a submodule $N \subseteq M$ s.t. $M$ can be expressed by the internal direct sum $M = i(M') \oplus N$; and $p$ induces an isomorphism $N \simeq M''$.
    \end{enumerate}
    Such a short exact sequence is a \underline{split exact sequence}.
\end{proposition}

\begin{proof}
    It suffices to give the equivalence between i) and iii), as for ii) it is similar.
    \begin{itemize}
        \item \emph{i) $\Rightarrow$ iii)}. Let $N = \ker q$. Check that this gives an internal direct sum:
            \begin{itemize}
                \item $N \cap i(M') = 0$. Let $x \in i(M') \cap N$. $x \in N$ implies $q(x) = 0$, while $q \circ i = \Id_{M'}$ implies that $x = 0$.
                \item $N + i(M') = M$. Notice $v - i \circ q(v) \in \ker q$, and by inspection $i\circ q(v) \in \im i$.
            \end{itemize}
            By the first isomorphism theorem, $\im i = \ker p$ implies $M / \im i \simeq N \simeq M''$.
        \item \emph{iii) $\Rightarrow$ i)}. Define $i: i(M') \oplus N \to i(M') \simeq M'$ since $i$ is injective.
    \end{itemize}
\end{proof}

\begin{remark}
    Generally short exact sequences do not split. A counterexample is
    \[
        0 \too \Z \overset{i}{\too} 2\Z \overset{p}{\too} \Z/2\Z \too 0
    \]
    where $i(\Z) \simeq \Z$, but the map cannot be extended properly to the whole $\Z$. If $R$ is a field, then all short exact sequences split as one can complete a basis in a vector space; and subspaces spanned by a subset of a basis is always a direct summand of the whole space.
\end{remark}

The following present a common technique known as ``diagram chasing'':

\begin{proposition}[The 5-Lemma]
    Consider the following diagram, with blocks commute and rows exact:
    \begin{figure}[htbp]
        \centering
        \begin{tikzcd}
            A_1 \arrow[d, "f_1"] \arrow[r, "u_1"] & A_2 \arrow[d, "f_2"] \arrow[r, "u_2"] & A_3 \arrow[d, "f_3"] \arrow[r, "u_3"] & A_4 \arrow[d, "f_4"] \arrow[r, "u_4"] & A_5 \arrow[d, "f_5"] \\
            B_1 \arrow[r, "v_1"] & B_2 \arrow[r, "v_2"] & B_3 \arrow[r, "v_3"] & B_4 \arrow[r, "v_4"] & B_5 \\
        \end{tikzcd}
    \end{figure}
    \begin{enumerate}
        \item If $f_2, f_4$ are injective, $f_1$ is surjective, then $f_3$ is injective.
        \item If $f_2, f_4$ are surjective, $f_5$ is injective, then $f_3$ is surjective.
        \item (Combining i) and ii)) If $f_1, f_2, f_4, f_5$ are all isomorphisms, then $f_3$ is an isomorphism.
    \end{enumerate}
\end{proposition}

\begin{proof}
    The argument is symmetric, so it suffices to prove the first one. $f_3$ is injective if and only if $f_3(b) = 0 \implies b = 0$. Following the steps:
    \begin{itemize}
        \item Consider the third square. $v_3 \circ f_3(b) = v_3(0) = 0$, giving $f_4 \circ u_3(b) = 0$. $f_4$ being injective implies that $u_3(b) = 0$. 
        \item Consider the second square. The top row being exact implies that $b \in \im u_2$, i.e. there exists some $c \in A_2$ s.t. $u_2(c) = b$. Commutativity gives that $v_2 \circ f_2(c) = 0$, i.e. $c' := f_2(c) \in \ker v_1$.
        \item Consider the first square. The bottom row being exact implies that there exists some $d' \in B_1$ s.t. $v_1(d') = c'$. Since $f_1$ is surjective, there exists $d \in A_1$ s.t. $f_1(d) = d'$. For the diagram to commute, it is required that $u_1(d) = c$. But this indicates that $c \in \im u_1$, i.e. $c \in \ker u_2$, which gives $b = u_2(c) = 0$.
    \end{itemize}
\end{proof}

% TODO Add Snake Lemma

\begin{definition}
    Let $R$ and $S$ be rings, and $F: \catlmod{R} \to \catlmod{S}$ is an additive functor. Then $F$ is \textbf{exact} if for all short exact sequences of $R$-modules $0 \too M' \too M \too M'' \too 0$, the corresponding sequence after applying $F$ is also exact.
\end{definition}

\begin{proposition}
    $F$ is exact if and only if for all exact sequence $A \overset{f}{\too} B \overset{g}{\too} C$, $F(A) \overset{f}{\too} F(B) \overset{g}{\too} F(C)$ is also exact.
\end{proposition}

\begin{proof}
    Proceed by showing implication in two directions:
    \begin{itemize}
        \item[$\Leftarrow$:] This holds by definition, where $f$ is injective and $g$ is surjective.
        \item[$\Rightarrow$:] Consider the following short exact sequences:
        \[
        \begin{array}{ll}
            (1):\qquad\qquad & 0 \too \ker f \too A \overset{\alpha_1}{\too} \im f \too 0 \\
            (2):\qquad\qquad & 0 \too \ker g \overset{\alpha_2}{\too} B \overset{\beta_1}{\too} \im g \too 0 \\
            (3):\qquad\qquad & 0 \too \im g \overset{\beta_2}{\too} C \too \coker g \too 0
        \end{array}
        \]
        where $\im f = \ker g$ as the sequence given is exact. These by construction are all short exact sequences, where applying $F$ gives also short exact sequences. Combining gives the sequence which is still exact after applying $F$:
        \[
            A \overset{\alpha_1}{\too} \im f \overset{\alpha_2}{\too} B \overset{\beta_1}{\too} \im g \overset{\beta_2}{\too} C
        \]
        where $\alpha_1, \beta_1$ are surjective; and $\alpha_2, \beta_2$ are injective. What we want to show is $\im F(f) = \ker F(g)$. Since $\alpha_1$ is surjective, $\im F(f) = \im F(\alpha_2)$; and since $\beta_2$ is injective, $\ker F(g) = \ker F(\beta_1)$. From the result of (2) after applying $F$, we have $\im F(\alpha_2) = \ker F(\beta_1)$.
    \end{itemize}
\end{proof}

\begin{remark}
    ``One-sided'' exact sequences can be understood functorially:
    \begin{itemize}
        \item Given exact sequence $0 \too M' \overset{i}{\too} M \overset{p}{\too} M''$ is the same as saying that $i$ is injective; and $M'$ is the kernel of $p$.
        \item Similarly, given exact sequence $M' \overset{i}{\too} M \overset{p}{\too} M'' \too 0$ is the same as saying that $p$ is surjective; and $M''$ is the cokernel of $i$.
    \end{itemize}
\end{remark}

\begin{definition}
    Just as the remark, one could consider exact functors only on one side. $F: \catlmod{R} \to \catlmod{S}$ is \textbf{left exact} if for all exact sequence $0 \too A \too B \too C$, the sequence $0 \too F(A) \too F(B) \too F(C)$ is also exact; and the definition is symmetric for right exact functors. Notice that since $F$ is an additive functor, $F(0) = 0$ (as zero morphisms are mapped to zero morphisms). 
\end{definition}

\begin{proposition}\label{prop:Hom(M,-) exact}
    Let $M$ be a left $R$-module. Then functor $F = \Hom_R(M, -): \catlmod{R} \to \catlmod{S}$ is exact; and the coverse is also true, i.e. if $0 \too \Hom_R(M, A) \too \Hom_R(M, B) \too \Hom_R(M, C)$ is exact, then $0 \too A \too B \too C$ is exact.
\end{proposition}

\begin{proof}
    What we want to show is that if the sequence $0 \too A \too B \too C$ is exact, then the corresponding sequence $0 \too \Hom(M, A) \too \Hom(M, B) \too \Hom(M, C)$ is exact:
    \begin{figure}[htbp]
        \centering
        \begin{tikzcd}
            0 \arrow[r] & A \arrow[r, "f"] & B \arrow[r, "g"] & C \\
            & & M \arrow[u, "v"] \arrow[lu, dashed, "u"] & \\
        \end{tikzcd}
    \end{figure}
    The natural way to define the functor $F$ is via specifying $\Hom(M, A) \ni u \mapsto f \circ u$, $\Hom(M, B) \ni v \mapsto g \circ v$. Exactness follows from the universal property of kernel, where for all $v \in \Hom(M, B)$ s.t. $\Hom(M, C) \ni g \circ v = 0$, it factors uniquely through $f$. 

    For the converse, take $M = A$, which gives $F(g) \circ F(f) = 0$, i.e. $F(f) = \ker F(g)$. But $F(f) = f$ by construction, which gives the exactness.  
\end{proof}

\begin{remark}
    The symmetric argument is also true, via applying the universal property of cokernel. 
\end{remark}

\begin{proposition}
    If $M$ is an $S$-$R$ bimodule, then $M \tensor_R -$ is right exact. 
\end{proposition}

\begin{proof}
    From the right exact version of \ref{prop:Hom(M,-) exact}, it suffices to prove that for left $R$-module $N$ the sequence
    \[
        \Hom(M \tensor A, N) \too \Hom(M \tensor B, N) \too \Hom(M \tensor C, N) \too 0
    \]
    is exact. First make a parenthesis on the generalization of the adjoint property of extension and restriction of scalars:

    \begin{parenthesis}[Prop 4.4 \lbrack c6\rbrack]
        Let $M$ be an $S$-$R$ bimodule, $C$ a left $R$-module, and $N$ a left $S$-module, then there is a functorial isomorphism
        \[
            \Hom_S(M \tensor_R C, N) \simeq \Hom_R(C, \Hom_S(M, N))
        \]
    \end{parenthesis}

    \begin{proof}
        By the universal property of tensor product, it suffices to give every $R$-balanced map $f: M \times C \to N$ a map in \\ $\Hom_R(C, \Hom_S(M, N))$. Denote the isomorphism to be $F$. Let $F(\tilde{f}[u\tensor v \mapsto f(u, v)]) = v\mapsto (u \mapsto f(u, v))$. The inverse exists by inspection.
    \end{proof}

    Then apply the parenthesis and \ref{prop:Hom(M,-) exact} gives that it suffices to verify that $\Hom(M, A) \too \Hom(M, B) \too \Hom(M, C) \too 0$ is exact, which holds as $M$ is in particular a right $R$-module. 
\end{proof}

\begin{remark}
    Recall that the above isomorphism holds for all adjoint pairs $(F, G)$. Therefore, the proof applies as long as $F$ is left exact (for $G$ being right exact) or the converse holds. 
\end{remark}

In general, the above two functors and the contravariant is only left (right) exact instead of exact. Consider the short exact sequence:
\[
    0 \too \Z \too 2\Z \too \Z/2\Z \too 0
\]
\begin{itemize}
    \item Applying $- \tensor \Z/2\Z$ gives
    \[
        0 \too \Z \tensor \Z/2\Z \overset{i}{\too} 2\Z \tensor \Z/2\Z \overset{p}{\too} \Z/2\Z \tensor \Z/2\Z \too 0
    \]
    where $2\Z \tensor \Z/2\Z = 0$ and $\Z \tensor \Z/2\Z \simeq \Z/2\Z \tensor \Z/2\Z \simeq \Z/2\Z$, as the map being $R$-linear restricts that $f(0, 1) = f(1, 0) = f(0, 0)$. This implies that $i$ is not injective.
    \item Applying $\Hom_{\Z}(-, \Z/2\Z)$ gives
    \[
        0 \too \Hom_{\Z}(\Z, \Z/2\Z) \overset{i}{\too} \Hom_{\Z}(2\Z, \Z/2\Z) \overset{p}{\too} \Hom_{\Z}(\Z/2\Z, \Z/2\Z) \too 0
    \]
    where $\Hom_{\Z}(2\Z, \Z/2\Z) \ni h = 0$ as it is required that $h(1) + h(1) = \bar{1} + \bar{1} = 0$ which indicates that $p$ is not surjective. Similar situations appear in the contravariant case.
\end{itemize}

It is then of specific interest in which modules are the above functors exact.

\section{Flat/Projective/Injective Modules}

\end{document}