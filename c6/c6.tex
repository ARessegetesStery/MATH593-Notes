\documentclass{article}
\usepackage{../refalg}

\begin{document}
\Makepagesectionhead{MATH 593}{Introduction to Homological Algebra}{ARessegetes Stery}

\tableofcontents
\newpage

\def\too{\longrightarrow}
\def\Kom{\underline{\mathrm{Kom}}}
\def\comp#1{#1^{\circ}}

\section{Exactness}

\begin{definition}[Complex]
    A \textbf{Complex} of $R$-modules is a family of $R$-modules $(M_i)$ and $R$-linear maps $d_i: M_i \to M_{i+1}$ s.t. for all $i$, $d_{i+1} \circ d_i = 0$.
\end{definition}

\begin{remark}
    The followings are some specifications on the notations:
    \begin{itemize}
        \item The complex is often denoted by a chain
        \[
            \cdots \overset{d_{i-2}}{\too} M^{i-1} \overset{d_{i-1}}{\too} M^i \overset{d_{i}}{\too} M^{i+1} \overset{d_{i+1}}{\too} \cdots
        \]
        or a chain with indices on the bottom with $M^i = M_{-i}$.
        \item The complex extends to infinity in both ends. If the notation terminated on one side, all modules not written out are the trivial (the zero module).
    \end{itemize}
\end{remark}

\begin{remark}
    The definition of a complex is the same as stating that $\im d_{i} \subseteq \ker d_{i+1}$ for all $i$.
\end{remark}

\begin{definition}
    For a sequence $A \overset{f}{\too} B \overset{g}{\too} C$ where $f$ and $g$ are $R$-linear maps, it is \textbf{exact at $\mathbf{B}$} if the equality is reached in the remark above, i.e. $\im f = \ker g$. 

    A sequence is exact if it is exact at $A_i$ for all $i$. A complex is exact if it is exact everywhere.
\end{definition}

\begin{example}
    The sequence $0 \too A \overset{f}{\too} B$ is exact implies that $\ker f = \{0\}$, i.e. $f$ is injective. Similarly, $A \overset{g}{\too} B \too 0$ implies that $g$ is surjective.
\end{example}

\begin{definition}
    A \textbf{Short Exact Sequence (SES)} is an exact sequence
    \[
        0 \too M' \overset{i}{\too} M \overset{p}{\too} M'' \too 0
    \]
\end{definition}

\begin{proposition}
    Given a sequence $(\ast): 0 \too M' \overset{i}{\too} M \overset{p}{\too} M'' \too 0$, the followings are equvalent:
    \begin{enumerate}[label=\roman*)]
        \item $(\ast)$ is a short exact sequence.
        \item $i$ is injective, and for $q: M \to \coker i$, there exists a unique isomorphism $\theta$ s.t. $\theta \circ q = p$.
        \item $p$ is surjective, and for $j: M \to \ker p$, there exists a unique isomorphism $\eta$ s.t. $i = \eta \circ j$.
    \end{enumerate}
\end{proposition}

\begin{proof}
    It suffices to prove the equivalence between i) and ii), as the case with iii) is similar:
    \begin{itemize}
        \item \emph{i) $\Rightarrow$ ii)}. Apply the universal property of cokernel. Since $(\ast)$ is exact, $p \circ i = 0$, there exists a map $\theta$ s.t. the following diagram commutes.
        \begin{figure}[htbp]
            \centering
            \begin{tikzcd}
                M' \arrow[rr, "i"] & & M \arrow[rr, "p"] \arrow[rrdd, "q"] & & M'' \arrow[dd, <-, "\theta"] \\
                & & & & \\
                & & & & \coker i
            \end{tikzcd}
        \end{figure}
        The fact that $p$ is surjective, and the diagram should commute gives $\theta$ should be surjective. To prove that $\theta$ is injective, it suffices to verify that $\theta(b) = 0 \implies b = 0$ for $b \in \coker i$. Since $q$ by definition is surjective, there exists $a \in M$ s.t. $q(a) = b$. This gives $a \in \ker p = \im i$, which implies that $q(a) = 0$ as the cokernel is defined by $M/ \im i$.
        \item \emph{ii) $\Rightarrow$ i)}. Given that $\mu$ is an isomorphism and $i$ is injective, it suffices to verify that $p$ is surjective, and $\im i = \ker p$. $\mu$ being surjective implies that $p$ is surjective; and $\mu$ being an isomorphism implies that $\ker p = \ker q = \im i$. 
    \end{itemize}
\end{proof}

\begin{proposition}\label{prop:criterion for split}
    Given a short exact sequence $0 \too M' \overset{i}{\too} M \overset{p}{\too} M'' \too 0$, the following statements are equivalent:
    \begin{enumerate}[label=\roman*)]
        \item There exists $j: M \to M'$ s.t. $j \circ i = \Id_{M'}$
        \item There exists $q: M'' \to M$ s.t. $p \circ q = \Id_{M''}$
        \item There exists a submodule $N \subseteq M$ s.t. $M$ can be expressed by the internal direct sum $M = i(M') \oplus N$; and $p$ induces an isomorphism $N \simeq M''$.
    \end{enumerate}
    Such a short exact sequence is a \underline{split exact sequence}.
\end{proposition}

\begin{proof}
    It suffices to give the equivalence between i) and iii), as for ii) it is similar.
    \begin{itemize}
        \item \emph{i) $\Rightarrow$ iii)}. Let $N = \ker j$. Check that this gives an internal direct sum:
            \begin{itemize}
                \item $N \cap i(M') = \{0\}$. Let $x \in i(M') \cap N$. Then there exists $u \in M'$ s.t. $i(u) \in \ker j$, i.e. $j\circ i(u) = 0$. But this indicates that $u = 0$ as $j \circ i = \Id_{M'}$. Since $i$ is a morphism of modules, $i(0) = 0$, which indicates that the only element that is in both $i(M')$ and $N$ is 0.
                \item $N + i(M') = M$. Notice $v - i \circ j(v) \in \ker q$, and by inspection $i\circ j(v) \in \im i$.
            \end{itemize}
            By the first isomorphism theorem, $\im i = \ker p$ implies $M / \im i \simeq N \simeq M''$.
        \item \emph{iii) $\Rightarrow$ i)}. Define $j: i(M') \oplus N \to i(M') \simeq M'$ since $i$ is injective.
    \end{itemize}
\end{proof}

\begin{remark}
    Generally short exact sequences do not split. A counterexample is
    \[
        0 \too \Z \overset{i}{\too} 2\Z \overset{p}{\too} \Z/2\Z \too 0
    \]
    where $i(\Z) \simeq \Z$, but the inverse map cannot be extended properly to the whole ring $\Z$. If $R$ is a field, then all short exact sequences split as one can complete a basis in a vector space; and subspaces spanned by a subset of a basis is always a direct summand of the whole space.
\end{remark}

The following present a common technique known as ``diagram chasing'':

\begin{proposition}[The 5-Lemma]
    Consider the following diagram, with blocks commute and rows exact:
    \begin{figure}[htbp]
        \centering
        \begin{tikzcd}
            A_1 \arrow[d, "f_1"] \arrow[r, "u_1"] & A_2 \arrow[d, "f_2"] \arrow[r, "u_2"] & A_3 \arrow[d, "f_3"] \arrow[r, "u_3"] & A_4 \arrow[d, "f_4"] \arrow[r, "u_4"] & A_5 \arrow[d, "f_5"] \\
            B_1 \arrow[r, "v_1"] & B_2 \arrow[r, "v_2"] & B_3 \arrow[r, "v_3"] & B_4 \arrow[r, "v_4"] & B_5 \\
        \end{tikzcd}
    \end{figure}
    \begin{enumerate}
        \item If $f_2, f_4$ are injective, $f_1$ is surjective, then $f_3$ is injective.
        \item If $f_2, f_4$ are surjective, $f_5$ is injective, then $f_3$ is surjective.
        \item (Combining i) and ii)) If $f_1, f_2, f_4, f_5$ are all isomorphisms, then $f_3$ is an isomorphism.
    \end{enumerate}
\end{proposition}

\begin{proof}
    The argument is symmetric, so it suffices to prove the first one. $f_3$ is injective if and only if $f_3(b) = 0 \implies b = 0$. Following the steps:
    \begin{itemize}
        \item Consider the third square. $v_3 \circ f_3(b) = v_3(0) = 0$, giving $f_4 \circ u_3(b) = 0$. $f_4$ being injective implies that $u_3(b) = 0$. 
        \item Consider the second square. The top row being exact implies that $b \in \im u_2$, i.e. there exists some $c \in A_2$ s.t. $u_2(c) = b$. Commutativity gives that $v_2 \circ f_2(c) = 0$, i.e. $c' := f_2(c) \in \ker v_1$.
        \item Consider the first square. The bottom row being exact implies that there exists some $d' \in B_1$ s.t. $v_1(d') = c'$. Since $f_1$ is surjective, there exists $d \in A_1$ s.t. $f_1(d) = d'$. For the diagram to commute, it is required that $u_1(d) = c$. But this indicates that $c \in \im u_1$, i.e. $c \in \ker u_2$, which gives $b = u_2(c) = 0$.
    \end{itemize}
\end{proof}

% TODO Add Snake Lemma

\begin{definition}
    Let $R$ and $S$ be rings, and $F: \catlmod{R} \to \catlmod{S}$ is an additive functor. Then $F$ is \textbf{exact} if for all short exact sequences of $R$-modules $0 \too M' \too M \too M'' \too 0$, the corresponding sequence after applying $F$ is also exact.
\end{definition}

\begin{proposition}
    $F$ is exact if and only if for all exact sequence $A \overset{f}{\too} B \overset{g}{\too} C$, $F(A) \overset{f}{\too} F(B) \overset{g}{\too} F(C)$ is also exact.
\end{proposition}

\begin{proof}
    Proceed by showing implication in two directions:
    \begin{itemize}
        \item[$\Leftarrow$:] This holds by definition, where one can consider the particular case where $f$ is injective and $g$ is surjective.
        \item[$\Rightarrow$:] Consider the following short exact sequences:
        \[
        \begin{array}{ll}
            (1):\qquad\qquad & 0 \too \ker f \too A \overset{\alpha_1}{\too} \im f \too 0 \\
            (2):\qquad\qquad & 0 \too \ker g \overset{\alpha_2}{\too} B \overset{\beta_1}{\too} \im g \too 0 \\
            (3):\qquad\qquad & 0 \too \im g \overset{\beta_2}{\too} C \too \coker g \too 0
        \end{array}
        \]
        where $\im f = \ker g$ as the sequence given is exact. These by construction are all short exact sequences, where applying $F$ gives also short exact sequences. Combining gives the sequence which is still exact after applying $F$:
        \[
            A \overset{\alpha_1}{\too} \im f \overset{\alpha_2}{\too} B \overset{\beta_1}{\too} \im g \overset{\beta_2}{\too} C
        \]
        where $\alpha_1, \beta_1$ are surjective; and $\alpha_2, \beta_2$ are injective. What we want to show is $\im F(f) = \ker F(g)$. Since $\alpha_1$ is surjective, $\im F(f) = \im F(\alpha_2)$; and since $\beta_2$ is injective, $\ker F(g) = \ker F(\beta_1)$. From the result of (2) after applying $F$, we have $\im F(\alpha_2) = \ker F(\beta_1)$.
    \end{itemize}
\end{proof}

\begin{remark}
    ``One-sided'' exact sequences can be understood functorially:
    \begin{itemize}
        \item Given exact sequence $0 \too M' \overset{i}{\too} M \overset{p}{\too} M''$ is the same as saying that $i$ is injective; and $M'$ is the kernel of $p$.
        \item Similarly, given exact sequence $M' \overset{i}{\too} M \overset{p}{\too} M'' \too 0$ is the same as saying that $p$ is surjective; and $M''$ is the cokernel of $i$.
    \end{itemize}
\end{remark}

\begin{definition}
    Just as in the remark, one could consider exact functors only on one side. $F: \catlmod{R} \to \catlmod{S}$ is \textbf{left exact} if for all exact sequence $0 \too A \too B \too C$, the sequence $0 \too F(A) \too F(B) \too F(C)$ is also exact; and the definition is symmetric for right exact functors. Notice that since $F$ is an additive functor, $F(0) = 0$ (as zero morphisms are mapped to zero morphisms). 
\end{definition}

\begin{proposition}\label{prop:Hom(M,-) left exact}
    Let $M$ be an $R$-$S$ bimodule. Then functor $F = \Hom_R(M, -): \catlmod{R} \to \catlmod{S}$ is left exact; and the converse is also true, i.e. if $0 \too \Hom_R(M, A) \too \Hom_R(M, B) \too \Hom_R(M, C)$ is exact, then $0 \too A \too B \too C$ is exact.
\end{proposition}

\begin{proof}
    What we first want to show is that if the sequence $0 \too A \too B \too C$ is exact, then the corresponding sequence $0 \too \Hom(M, A) \too \Hom(M, B) \too \Hom(M, C)$ is exact:
    \begin{figure}[htbp]
        \centering
        \begin{tikzcd}
            0 \arrow[r] & A \arrow[r, "f"] & B \arrow[r, "g"] & C \\
            & & M \arrow[u, "v"] \arrow[lu, dashed, "u"] & \\
        \end{tikzcd}
    \end{figure}
    The natural way to define the functor $F$ is via specifying $\Hom(M, A) \ni u \mapsto f \circ u$, $\Hom(M, B) \ni v \mapsto g \circ v$. Exactness follows from the universal property of kernel, where for all $v \in \Hom(M, B)$ s.t. $\Hom(M, C) \ni g \circ v = 0$, it factors uniquely through $f$. Furthermore, since $F$ is an additive functor, it preserves injectivity (via considering elements in $\catsets$), which indicates that $F(f)$ is also injective.

    For the converse, take $M = A$. This gives the exact sequence $0 \too \Hom (A, A) \overset{\alpha}{\too} \Hom(A, B) \overset{\beta}{\too} \Hom(A, C)$. The natural ways to define the map is via specifying $\alpha = f \circ -$, $\beta = g \circ -$. Observe that the exactness of the sequence gives $\beta \circ \alpha = g \circ f $ \\ $\circ - = 0$, which indicates that $g \circ f = 0$ by associativity. $\alpha$ being injective follows directly from the fact that $f$ is injective.
\end{proof}

\begin{remark}
    The dual argument is also true, via applying the universal property of cokernel. That is, the functor $\Hom_R(-, M)$ is right exact. The direction of exactness reverses as the functor is contravariant. 
\end{remark}

\begin{proposition}\label{prop:tensor product functor right exact}
    If $M$ is an $S$-$R$ bimodule, then $M \tensor_R -$ is right exact. 
\end{proposition}

\begin{proof}
    From the right exact version of Proposition \ref{prop:Hom(M,-) left exact}, it suffices to prove that for left $R$-module $N$ the sequence
    \[
        \Hom(M \tensor A, N) \too \Hom(M \tensor B, N) \too \Hom(M \tensor C, N) \too 0
    \]
    is exact. First make a parenthesis on the generalization of the adjoint property of extension and restriction of scalars:

    \begin{parenthesis}[Prop 4.4 \lbrack c6\rbrack]
        Let $M$ be an $S$-$R$ bimodule, $C$ a left $R$-module, and $N$ a left $S$-module, then there is a functorial isomorphism
        \[
            \Hom_S(M \tensor_R C, N) \simeq \Hom_R(C, \Hom_S(M, N))
        \]
    \end{parenthesis}

    \begin{proof}
        By the universal property of tensor product, it suffices to give every $R$-balanced map $f: M \times C \to N$ a map in \\ $\Hom_R(C, \Hom_S(M, N))$. Let the isomorphism be $F$ defined via $F(\tilde{f}[u\tensor v \mapsto f(u, v)]) = v\mapsto (u \mapsto f(u, v))$. The inverse exists by inspection. It is well-defined as one can consider the map $u \times v \mapsto f(u, v)$; and use the universal property of tensor product.
    \end{proof}

    Then apply the parenthesis and Proposition \ref{prop:Hom(M,-) left exact} gives that it suffices to verify that $0 \too \Hom(M, C) \too \Hom(M, B) \too \Hom(M, A)$ is exact, which holds as $M$ is in particular a right $R$-module; and in this case the functor $\Hom_R(-, \Hom_S(M, N))$ is a contravariant functor. Recall that for a contravariant functor $\Hom(-, N)$, if the sequence $A \too B \too C \too 0$ is exact, then $0 \too \Hom(C, N) \too \Hom(B, N) \too \Hom(A, N)$ is exact. This finishes the proof.
\end{proof}

\begin{remark}
    Recall that the above isomorphism holds for all adjoint pairs $(F, G)$. Therefore, the proof applies as long as $F$ is left exact (for $G$ being right exact) or the converse holds. 
\end{remark}

In general, the above two functors and the contravariant is only left (right) exact instead of exact. Consider the short exact sequence:
\[
    0 \too \Z \too 2\Z \too \Z/2\Z \too 0
\]
\begin{itemize}
    \item Applying $- \tensor \Z/2\Z$ gives
    \[
        0 \too \Z \tensor \Z/2\Z \overset{i}{\too} 2\Z \tensor \Z/2\Z \overset{p}{\too} \Z/2\Z \tensor \Z/2\Z \too 0
    \]
    where $2\Z \tensor \Z/2\Z = 0$ and $\Z \tensor \Z/2\Z \simeq \Z/2\Z \tensor \Z/2\Z \simeq \Z/2\Z$, as the map being $R$-linear restricts that $f(0, 1) = f(1, 0) = f(0, 0)$. This implies that $i$ is not injective.
    \item Applying $\Hom_{\Z}(-, \Z/2\Z)$ gives
    \[
        0 \too \Hom_{\Z}(\Z, \Z/2\Z) \overset{i}{\too} \Hom_{\Z}(2\Z, \Z/2\Z) \overset{p}{\too} \Hom_{\Z}(\Z/2\Z, \Z/2\Z) \too 0
    \]
    where $\Hom_{\Z}(2\Z, \Z/2\Z) \ni h = 0$ as it is required that $h(1) + h(1) = \bar{1} + \bar{1} = 0$ which indicates that $p$ is not surjective. Similar situations appear in the contravariant case.
\end{itemize}

It is then of specific interest in which modules are the above functors exact.

\section{Flat, Projective, and Injective Modules}

\begin{definition}
    Let $R$ be a ring, and $M$ a left $R$-module. Then:
    \begin{itemize}
        \item $M$ is a \textbf{flat module} if $- \tensor_R M$ is an exact functor.
        \item $M$ is a \textbf{projective module} if $\Hom_R(M, -)$ is an exact functor.
        \item $M$ is an \textbf{injective module} if $\Hom_R(-, M)$ is an exact functor.
    \end{itemize}
\end{definition}

\begin{remark}\label{rmk:diff def for projective module}
    By comparing with the results obtained in the propositions aforementioned (Proposition \ref{prop:Hom(M,-) left exact}, \ref{prop:tensor product functor right exact}), it is clear what is further required by the definitions:
    \begin{itemize}
        \item By Proposition \ref{prop:tensor product functor right exact}, a module is flat if and only if for all injective maps $M_1 \to M_2$, the corresponding map \\ $M_1 \tensor M \to M_2 \tensor M$ is injective.
        \item By Proposition \ref{prop:Hom(M,-) left exact}, a module is projective if and only if for all surjective maps $M_1 \to M_2$, the corresponding map \\ $\Hom_R(M, M_1) \to \Hom_R(M, M_2)$ is surjective; or the corresponding map is injective if the functor is contravariant.
    \end{itemize}
\end{remark}

\begin{remark}\label{rmk:Alt def for projective module}
    The fact that Hom functors in the remark above is surjective is the same as the following definition (for projective modules): $M$ is a projective $R$-module if and only if for all morphism of $R$-modules $g: M \to V$ with module $U$ for which there exists a surjective morphism $f: U \to V$, there exists a morphism of $R$-modules $h: M \to U$ s.t. $g = f \circ h$; that is, making the following diagram commute:
    \begin{figure}[htbp]
        \centering
        \begin{tikzcd}
            & & U \arrow[dd, two heads, "f"] \\
            & & \\
            M \arrow[rr, "g"] \arrow[rruu, dashed, "h"] & & V
        \end{tikzcd}
    \end{figure}

    In plain words, there exists an embedding into some module (for example, free modules) that could ``project'' surjectively to $V$. This embedding (injection) definitely needs not be unique, as for example one could always embed the projective module to a free module with higher rank. 
    
    This definition is indeed equivalent with the previous one, as for a surjective morphism $f: U \to V$, for any $g: M \to V$ there exists some $h: M \to U$ s.t. the diagram commute. This indicates that the morphism $\Hom(M, U) \to \Hom(M, V)$ is a surjection.
    
    The dual result holds also for injective modules: for all $R$-module $M$ being injective, it is equivalent to state that for any morphism $g: M \to V$, there exists some $R$-module $U$ with an injection $f: V \to U$ and some morphism $h: U \to M$ s.t. $g = h \circ f$, i.e. the following diagram commutes:
    \begin{figure}[htbp]
        \centering
        \begin{tikzcd}
            M \arrow[rrdd, <-, "g"] \arrow[rr, <-, dashed, "h"] & & U \\
            & & \\
            & & V \arrow[uu, hookrightarrow, "f"]
        \end{tikzcd}
    \end{figure}
\end{remark}

\begin{proposition}\label{prop:criterion for projective module}
    Let $M$ be a left $R$-module. Then the following statements are equivalent:
    \begin{enumerate}[label=\roman*)]
        \item $M$ is a projective $R$-module.
        \item $M$ is a direct summand of a free $R$-module. That is, there exists a free $R$-module $F$ and an $R$-module $N$ s.t. $F \simeq M \oplus N$.
        \item Let $P$ be an $R$-module, and $F$ be a free $R$-module. Then every short exact sequence $0 \too P \too F \too M \too 0$ is split exact.
    \end{enumerate}
\end{proposition}

\begin{proof}\ 
    \begin{itemize}
        \item \emph{i) $\implies$ iii)}. Apply the definition given in Remark \ref{rmk:Alt def for projective module}. $M$ being an $R$-module implies that it admits a system of generators, namely there exists some $(g_i)_{i\in I}$ s.t. $M = (g_i)_{i\in I}$. Then there exist $F = R^{(I)}$ where there exists a surjection $f: F \to M, f(e_i) = g_i$. Let $g: M \to M$ be the identity map. By the alternative definition there exists some $h: M \to F$ s.t. $f \circ h = g = \Id_M$. By Proposition \ref{prop:criterion for split} this indicates that the sequence of interest splits.
        \item \emph{iii) $\implies$ ii)}. Consider the projection $p: F \to M$ where $M = (g_i)_{i\in I}$ and $F = R^{(I)}$. Therefore, the sequence
        \[
            0 \too \ker p \overset{i}{\too} F \too M \too 0
        \]
        is short exact, which gives $F \simeq i(\ker p) \oplus M = \ker p \oplus M$.
        \item \emph{ii) $\implies$ i)}. Consider $F$ as a projective module. This is true as $F$ can be trivially embedded to itself. It then suffices to prove that if $F$ is projective and $F \simeq M \oplus N$, then $M$ is projective. By Remark \ref{rmk:diff def for projective module}, it suffices to prove that $\Hom_R(M, U) \to \Hom_R(M, V)$ is surjective for all $U \to V$ surjective. $F$ is projective indicates that $\Hom_R(F, U) \to \Hom_R(F, V)$ is surjective. By the fact that $\Hom_R(F, U) \simeq \Hom_R(M, U) \oplus \Hom_R(N, U)$, $\Hom_R(M, U) \to \Hom_R(M, V)$ is surjective.
    \end{itemize}
\end{proof}

\begin{corollary}
    The category of $\catlmod{R}$ has enough projective modules. In particular, for $M = (g_i)_{i\in I}$ one can take $F = R^{(I)}$. 
\end{corollary}

Similarly, we would like to prove that there are ``enough'' injective objects in $\catlmod{R}$:

\begin{theorem}\label{thm:enough injectives}
    The category of $\catlmod{R}$ has enough injective objects. That is, for all $R$-module $M$, there exists an injective embedding $M \hookrightarrow Q$ where $Q$ is an injective module.
\end{theorem}

\begin{proposition}[Baer]\label{prop:Baer}
    Let $M$ be an $R$-module. Then $M$ is injective if and only if $\Hom_R(R, M) \to \Hom_R(I, M)$ is surjective for all left ideals $I \subseteq R$. That is, for all $R$-linear maps $f: I \to M$, there exists some $u\in M$ s.t. $f(x) = xu$ for all $x\in I$ (and this gives an extension into $R$).
\end{proposition}

\begin{proof}
    The implication from $M$ being injective to the condition follows directly from the definition of injective modules, where one considers the map $I \hookrightarrow R$ to be the injection. For the other direction, consider the following construction:

    Let $i: M_1 \hookrightarrow M_2$ be an injection, where $M_1, M_2$ are $R$-modules; and let $f: M_1 \to M$ be an $R$-linear map. It suffices to show that there exists some $g: M_2 \to M$ s.t. $g \circ i = f$. Consider the family of modules $N_i$ and corresponding maps $h_i$ s.t. $M_1 \subseteq N_i \subseteq M_2$ for all $i$; and $h_i|_{M_1} = f$. Define the partial order $(N, h) \leq (N', h')$ if and only if $N \subseteq N'$, and $h'|_N = h$. Notice that such family is non-empty, as in particular $M_1$ is in the family. Further it is bounded above by $M_2$, which allows us to apply Zorn's Lemma to retrieve a maximal element $(\bar{N}, \bar{h})$. Handle the cases respectively:
    \begin{itemize}
        \item $\bar{N} = M_2$. Then letting $g = \bar{h}$ finishes the proof.
        \item $\bar{N} \neq M_2$. Proceed to show that this map can be further extended, which is a contradiction.
        
        By the inequality there exists some $x \in M_2 - \bar{N}$. Consider $I = \{a \in R \mid ax \in N\}$ which is the submodule of $M_2$ generated by $x$. Consider the morphism of $R$-modules $f'$ where $f'(a) = \bar{h}(ax)$. By hypothesis there exists $u\in M$ s.t. $f'(x) = xu$ for all $x \in I$. Notice that $\bar{h} \circ \bar{i} = f$ where $\bar{i}: M_1 \hookrightarrow \bar{N}$ is the injection. Define the map $\tilde{h}: \bar{N} + I = \bar{N} + Rx \to M , \tilde{h}(a + vx) = \bar{h}(a) + vu$ for $a \in \bar{N}$ and $v\in N$. This is well-defined, as for all $v$ s.t. $vx \in \bar{N}$, $\bar{h}(vx) = h(vx) = f'(v) = vu$ by hypothesis. This indicates that $\bar{N}$ is not maximal as the map can be extended to $\bar{N} + Rx$, which is a contradiction.
    \end{itemize}
\end{proof}

\begin{definition}
    Let $M$ be an $R$-module. Then it is \textbf{divisble} if and only if for all $u \in M$, $n \in \Z_{\geq 0}$, there exists $v \in M$ s.t. $nv = u$ where multiplication by $n \in \Z_{\geq 0}$ is adding $n$ copies of the elements to itself.
\end{definition}

\begin{corollary}
    If $R = \Z$, then $M$ being an $R$-module is injective if and only if it is divisible.
\end{corollary}

\begin{proof}
    If $M$ is divisible, then for all $n\in \Z_{\geq 0}$, $u\in M$ there exists $v \in M$ s.t. $u = nv$. That is, for all $f: I \to M$ with $n \in I \subseteq \Z$, there exists $v \in M$ s.t. $f(n) = u = nv$. This gives the criterion in Proposition \ref{prop:Baer}. The converse holds as the converse holds in the proposition.
\end{proof}

\begin{proof}[Proof of Theorem \ref{thm:enough injectives}]
    First prove the theorem with restriction $R = \Z$. From the the previous remark it is clear that $\Z$-modules are injective if and only if it is divisible; and $\Q$ as a $\Z$ module is divisible. Consider the canonical projection $\pi: \Z^{(I)} \to M$ where bases are mapped to generators. Then $M \simeq \Z^{(I)}/\ker \pi$, which embeds into $\Q^{(I)}/\ker \pi$. Since $\ker\pi$ is a submodule of $\Z^{(I)}$, $\Q^{(I)}/\ker \pi$ is divisible and thus injective; which satisfies the condition of interest.

    Now consider the general case by reducing to the case of $\Z$-modules. As $\Z$ modules, there exists an injection $h: M \hookrightarrow Q$, where $Q$ is an injective module. $Q$ being injective indicates that the functor $\Hom_{\Z}(-, Q)$ is exact (on $R$-modules). But notice that by applying the adjoint property this gives
    \[
        \Hom_{\Z}(-, Q) \simeq \Hom_{\Z}(R \tensor_R -, Q) \simeq \Hom_{R}(-, \Hom_{\Z}(R, Q))
    \] 
    which indicates that the functor $\Hom_{R}(-, \Hom_{\Z}(R, Q))$ is exact, i.e. $\Hom_{\Z}(R, Q)$ is an injective module. It then suffices to give an injective map from $M$ to $\Hom_{\Z}(R, Q)$, which is given by $u \mapsto (x \mapsto (h(xu)))$, where $u \in M$, and $x \in R$. It is well-defined and bilinear by inspection.
\end{proof}

The followings turn to the discussion of flat modules:

\begin{remark}
    Since the functor $- \tensor M$ for a given $R$-module $M$ is right exact, $M$ is flat if and only if for all injective maps $M_1 \to M_2$, $M_1 \tensor M \to M_2 \tensor M$ is also injective. 
\end{remark}

\begin{proposition}
    Given a family of left $R$-modules $(M_i)_{i \in I}$, $\bigoplus_{i\in I} M_i$ is flat if and only if $M_i$ is flat for all $i$.
\end{proposition}

\begin{proof}
    Since direct sum commutes with tensor product, given any injective $R$-linear map $N_1 \hookrightarrow N_2$, we have the following commutative diagram:
    \begin{figure}[htbp]
        \centering
        \begin{tikzcd}
            N_1 \tensor (\bigoplus_{i\in I} M_i) \arrow[d, "\simeq"] \arrow[r, hookrightarrow, "f"] & N_2 \tensor (\bigoplus_{i\in I} M_i) \arrow[d, "\simeq"] \\
            \bigoplus_{i\in I} ( N_1 \tensor M_i) \arrow[r, hookrightarrow, "f'"] & \bigoplus_{i\in I} ( N_2 \tensor M_i)
        \end{tikzcd}
    \end{figure}
    Notice $f'$ is an injection if and only if $f_i'$ is an injection for all $i$ by the universal property of coproduct; and $f$ is an injection if and only if $f'$ is an injection, as the diagram should commute.
\end{proof}

\begin{corollary}
    Projective modules are flat.
\end{corollary}

\begin{proof}
    By Proposition \ref{prop:criterion for projective module}, an $R$-module $M$ is projective if and only if there exists $R$-modules $F$ and $N$ where $F$ is free s.t. $F \simeq M \oplus N$. By the previous proposition, it suffices to prove that $F$ is flat. Since $F \simeq \oplus_{i \in I} R^{(I)}$, it suffices to prove that $R$ is flat. This is indeed the case as $- \tensor_R R = \Id_R$, which is trivially exact.
\end{proof}

\begin{definition}
    Let $R$ be a commutative ring, and $\varphi: R \to S$ specifies an $R$-algebra. Then $S$ is a \textbf{flat $\mathbf{R}$-algebra} if it is flat as an $R$-module. 
\end{definition}

\begin{remark}
    This implies that the extension of scalar functor is exact, as by the fact that tensor product should be $R$-balanced, multiplication is indeed scalar multiplication on the $R$-module itself, which preserves injection. 
\end{remark}

\begin{example}
    Consider the following flat structures:
    \begin{itemize}
        \item $R \to R[x_1, \cdots, x_n]$ is a flat $R$-algebra, as $R[x_1, \cdots, x_n]$ has a free $R$-module structure, where the basis is all monomials.
        \item Let $S \subseteq R$ be a multiplicative system. Then $R \to S^{-1}R$ is a flat $R$-algebra. It suffices to verify that $S^{-1}R \tensor -$ is exact. Notice $S^{-1}R \tensor M \simeq S^{-1}M$, this is the same as stating that $S^{-1}(-)$ is exact. 
        
        Since $N \subseteq M \implies S^{-1}N \subseteq S^{-1}M$, $S^{-1}(-)$ preserves injections; and since $S^{-1}(M/N) \simeq S^{-1}M/ S^{-1}N$, $S^{-1}(-)$ preserves surjection. Thus verifies the exactness and flatness.
    \end{itemize}
\end{example}

\begin{proposition}\label{prop:free iff projective in local ring}
    Let $(R, \mathfrak{m})$ be a local Noetherian ring, and $M$ a finitely generated $R$-module. Then $M$ is projective if and only if $M$ is free. 
\end{proposition}

\begin{proof}
    It suffices to verify that $M$ is free if it is projective. By Nakayama's Lemma, $(u_1, \cdots, u_m)$ forms a minimal system of generators of $M$ if and only if $(\bar{u}_1, \cdots, \bar{u}_m)$ forms a basis in $M/\mathfrak{m}$, This is indeed the case, as denoting $N = (u_1, \cdots, u_m)$, this gives $N + \mathfrak{m}M = M$, which indicates that $N = M$. Minimality is given by the minimality of cardinality of basis. Choose $F = R^m$ with $\varphi: F \to M$ s.t. $\varphi(e_i) = u_i$ for all $i$. Since $M$ is projective, consider the short exact sequence that splits:
    \[
        0 \too K \too F \overset{\varphi}{\too} M \too 0
    \]
    To prove that $M$ is free, it suffices to show that $K = 0$. Note that if $(a_1, \cdots, a_m) \in K$, then $\sum_{i=1}^m a_iu_i = 0 \implies \sum_{i=1}^m \bar{a}_i \bar{u}_i = 0 \implies \bar{a}_i = 0 \implies a_i \in \mathfrak{m}$ for all $i$ since $\bar{u}_i$s give a basis. That is, $K \subseteq \mathfrak{m} R^m$. Now apply the functor $-\tensor R/\mathfrak{m}$. This is an additive functor, which preserves morphisms as it acts as a group homomorphism, i.e. split exact sequences remain split exact after applying the functor. This gives the sequence
    \[
        0 \too K/\mathfrak{m}K \overset{\alpha}{\too} F/\mathfrak{m}F \too M/\mathfrak{m}M \too 0
    \] 
    which is also split. Since $K \subseteq \mathfrak{m} R^m$, $K \subseteq \mathfrak{m}F$, which indicates that $\alpha = 0$. Since $\alpha$ is injective as the sequence is exact, $K / \mathfrak{m}K = 0 \implies K = 0$. The sequence being split exact gives $F \simeq M \oplus K$, which implies $F \simeq M$. 
\end{proof}

\begin{corollary}
    Let $R$ be a Noetherian commutative ring, and $M$ a finitely generated $R$-module. Them $M$ is projective if and only if for all maximal ideals $p$ in $R$, $M_p$ is a free $R_p$ module.
\end{corollary}

\begin{proof}
    By Proposition \ref{prop:free iff projective in local ring}, it suffices to show that $M$ is projective if and only if for all maximal ideals $p$, the module $M_p$ is a projective $R_p$ module:
    \begin{enumerate}
        \item[$\Rightarrow$:] We prove a generalization of the statement. Let $\varphi: R \to S$ be a ring homomorphism, and $M$ be a projective $R$-module, then $M \tensor_R S$ is a projective $S$-module. Since $M$ is a projective $R$-module, there exists an $R$-module $N$ s.t. $R^{(I)} \simeq M \oplus N$. As tensor product commutes with direct sum, this gives the isomorphism $(M \tensor_R S) \oplus (N \tensor_R S) \simeq R^{(I)} \tensor_R S \simeq S^{(I)}$.
        \item[$\Leftarrow$:] Since for all maximal ideals $p$, $M_p$ is a projective $R_p$ module, by the definition of projective modules we have that for all $R$-modules $A$ and $B$ where there exists a surjective map $A \twoheadrightarrow B$, the morphism of $R$-modules $\Hom_{R_p} (M_p, A_p) \to \Hom_{R_p}(M_p, B_p)$ is surjective. Since tensor product is right exact, to show that $\Hom_R(M, A) \to \Hom_R(M, B)$ is surjective it suffices to show that the map $\Hom_R(M, A) \tensor R_p \to \Hom_R(M, B) \tensor R_p$ is surjective. The identification is shown via the following proposition:

        \begin{proposition}
            Let ring $S$ be a flat $R$-module, and $f: R \to S$ a ring homomorphism. Let $M$ and $N$ be $R$-modules. Then there is a canonical morphism of $S$-modules $\Hom_R(M, N) \tensor_R S \to \Hom_S(M \tensor_R S, N \tensor_R S)$ which is functorial w.r.t. $M$ and $N$. Further, if $R$ is Noetherian, and $M$ is finitely generated, then this is an isomorphism.
        \end{proposition}

        \begin{proof}
            Consider the morphism of $R$-modules $\Hom_R(M, N) \to \Hom_S(M \tensor_R S, N \tensor_R S)$, $f \mapsto f \tensor_R \Id_S$. Then by universal property of extension of scalars there exists a unique morphism of $S$-modules
            \begin{equation}\tag{$\ast$}
                \theta_M: \Hom_R(M, N) \tensor_R S \to \Hom_S(M \tensor_R S, N \tensor_R S), \qquad f \tensor a \mapsto a (f \tensor \Id_S)
            \end{equation}
            This is functorial w.r.t. both $M$ and $N$. Now regard this as a contravariant functor where $M$ is the argument: denote the morphism above to be $U(M) \to V(M)$. Observe the following facts:
            \begin{itemize}
                \item If $M = R$, then this is an isomorphism, as both sides are isomorphic to $N \tensor_R S$. 
                \item If $M$ is free, then this is also an isomorphism. This is a direct result of the fact above, and the fact that the functors are both additive, which commutes with direct sum.
            \end{itemize}
            Now consider the general case where $M$ is finitely generated over a Noetherian ring $R$. Then there exists a free module $F$ s.t. $F \to M$ is surjective, where $e_i$ is mapped to $u_i$ which gives a system of generators in $M$. $R$ is Noetherian indicates that $F$ is a Noetherian $R$-module, which gives that $K := \ker \varphi$ is finitely generated.

            Then similarly it is possible to take a free module $G$ where $G \to K$ is surjective. This gives an exact sequence $G \too F \too M \too 0$. (as $\im (G \to F) = \ker (F \to M)$). Applying $U$ and $V$ on the exact sequence gives the commutative diagram where the rows are exact, and the blocks commute:
            \begin{figure}[htbp]
                \centering
                \begin{tikzcd}
                    0 \arrow[r, dashed] \arrow[d, dashed] & 0 \arrow[r] \arrow[d, dashed] & U(M) \arrow[r] \arrow[d, "\theta_M"] & U(F) \arrow[r] \arrow[d, "\theta_F"] & U(G) \arrow[d, "\theta_G"] \\
                    0 \arrow[r, dashed] & 0 \arrow[r] & V(M) \arrow[r] & V(F) \arrow[r] & V(G)
                \end{tikzcd}
            \end{figure}
            The lines are indeed exact, as $S$ is flat on $R$, which gives that $- \tensor_S N$ is right exact, and $\Hom(-, N)$ is left exact. By the results on free modules, $\theta_F$ and $\theta_G$ are isomorphisms, which implies that $\theta_M$ is also an isomorphism by the 5-lemma. 
        \end{proof}

        This is indeed applicable, as in general localization preserves flatness. Consider $M$ a flat $R$-module, $S$ a multiplicative system in $R$, with an exact sequence $0 \too A \too B \too C \too 0$. To verify that this is true, it suffices to verify that the sequence $0 \too A \tensor_{S^{-1}R} S^{-1}M \too B \tensor_{S^{-1}R} S^{-1}M \too C \tensor_{S^{-1}R} S^{-1}M \too 0$ is exact. This is indeed true from the isomorphism 
        \[
            A \tensor_{S^{-1}R} S^{-1}M \simeq A \tensor_{S^{-1}R} (S^{-1}R \tensor_R M) \simeq (A \tensor_{S^{-1}R} S^{-1}R) \tensor_R M \simeq A \tensor_R M
        \]
        indicates that the sequence can be identified with $0 \too A \tensor_R M \too B \tensor_R M \too C \tensor_R M \too 0$, which is exact as $M$ is flat. In this case $R$ itself is flat as an $R$-module, which indicates that $R_p$ is also flat.
    \end{enumerate}

\end{proof}

\section{Complexes$^*$}

Recall that a complex is an infinite chain where the composition of any two successive morphisms gives zero. This section is devoted to further study this object.

First present another lemma which again uses the technique of ``chasing the diagram'':

\begin{proposition}[The Snake Lemma]
    Consider the diagram below, where all objects are in an additive category, where the rows are exact, blocks commute, and $0$ denotes the zero object: 
    \begin{figure}[htbp]
        \centering
        \begin{tikzcd}
            & A \arrow[r, "f"] \arrow[d, "a"] & B \arrow[r, "g"] \arrow[d, "b"] & C \arrow[r] \arrow[d, "c"] & 0 \\
            0 \arrow[r] & A' \arrow[r, "f'"] & B' \arrow[r, "g'"] & C' & 
        \end{tikzcd}
    \end{figure}

    Then there exists an exact sequence
    \[
        \ker a \too \ker b \too \ker c \overset{\delta}{\too} \coker a \too \coker b \too \coker c
    \]
    where $\delta$ is known as the \underline{connecting homomorphism}.
\end{proposition}

\begin{proof}[Sketch of Proof]
    The proof is done only for the case where the category is taken as $\catlmod{R}$.

    First define the map. Let $x \in \ker c$. $g$ being surjective implies that there exists some $y \in B$ s.t. $g(y) = x$. Denote $y' = b(y)$. By the fact that blocks commute, $g'(b(y)) = c(g(y)) = 0$, which implies that $y' \in \ker g'$. Since the second row is exact, $y'$ is in the image of $f'$, which implies that there exists some $z' \in A'$ s.t. $f'(z') = y'$. Define $\delta(x) = \bar{z'} \in A' / \im a$. This is a homomorphism as all maps used above are homomorphisms.

    Now check that this map is well-defined. Since $f'$ is injective by the exactness, it suffices to check that the choice of $z$ does not vary w.r.t. the choice of $y$. Suppose that there exists $y_1, y_2 \in B$ s.t. $(y_1 - y_2) \in \ker g$. By the exactness of the top row there exists $\tilde{z} \in A$ s.t. $f(\tilde{z}) = y_1 - y_2$ as $\ker g = \im f$. Let $z_1, z_2$ be the corresponding output where $y_1$ and $y_2$ are used, respectively. By the commutativity of the left block, together with the fact that $\delta$ is a homomorphism, $z_1' - z_2' = a(\tilde{z})$, which indicates that they have the same image in $\coker a$.

    Now check the exactness. Consider the expanded commutative diagram:
    \begin{figure}[htbp]
        \centering
        \begin{tikzcd}
            & \ker a \arrow[d, "a_1"] \arrow[r, "\alpha"] & \ker b \arrow[r, "\beta"] \arrow[d, "b_1"] & \ker c \arrow[r, dashed, "\delta"] \arrow[d, "c_1"] & \ \\
            & A \arrow[r, "f"] \arrow[d, "a"] & B \arrow[r, "g"] \arrow[d, "b"] & C \arrow[r] \arrow[d, "c"] & 0 \\
            0 \arrow[r] & A' \arrow[r, "f'"] \arrow[d, "a_2"] & B' \arrow[r, "g'"] \arrow[d, "b_2"] & C' \arrow[d, "c_2"] & \\
            \arrow[r, dashed, "\delta"] & \coker a \arrow[r, "\alpha'"] & \coker b \arrow[r, "\beta'"] & \coker c & \\
        \end{tikzcd}
    \end{figure}

    where the long exact sequence becomes
    \[
        \ker a \overset{\alpha}{\too} \ker b \overset{\beta}{\too} \ker c \overset{\delta}{\too} \coker a \overset{\alpha'}{\too} \coker b \overset{\beta'}{\too} \coker c
    \]
    Check the exactness respectively:
    \begin{itemize}
        \item At $\ker b$. We only need to show $\im \alpha = \ker \beta$. Since $a_1, b_1$ and $c_1$ are embeddings, this is just given by $\ker \beta = \ker b \cap \ker g$. 
        \item At $\ker c$. We need to show that $\im \beta = \ker \delta $. $x \in \im \beta$ implies that the corresponding $y'$ is zero. Then by injectivity of $f'$ this gives $\delta(x) = 0$, which gives $x \in \ker \delta$.
    \end{itemize}
    The exactness at rest two objects can be verified using a symmetric argument. 
\end{proof}

Now we turn to the subject of complexes.

\begin{definition}[Morphism of Complexes]
    Let $\comp{K}$ and $\comp{L}$ be complexes whose objects are $R$-modules. A morphism of complexes $f: \comp{K} \to \comp{L}$ is given by a family of $R$-linear maps $f^n: K^n \to L^n$ s.t. all the squares in the following diagram is commutative:
    \begin{figure}
        \centering
        \begin{tikzcd}
            \cdots \arrow[r] & K^n \arrow[r] \arrow[d, "f^n"] & K^{n+1} \arrow[r] \arrow[d, "f^{n+1}"] & \cdots \\
            \cdots \arrow[r] & L^n \arrow[r] & L^{n+1} \arrow[r] & \cdots \\
        \end{tikzcd}
    \end{figure}
\end{definition}

\begin{remark}
    Clearly commutative is transited through composition of morphisms. Since $\catlmod{R}$ gives an additive category structure, this is also a category $\Kom(\catlmod{R})$ (or $\Kom_R$).
\end{remark}

\begin{definition}[Cohomology]
    For a complex $\comp{K}: \cdots \overset{d^{n-1}}{\too} K^n \overset{d^n}{\too} K^{n+1} \too \cdots$, the \textbf{$n$-th cohomology} of $\comp{K}$ is defined as $H^n(K) := \ker(d^n)/ \im(d^{n-1})$.
\end{definition}

\begin{remark}
    Cohomology measures how far a point in the complex is from being exact. The complex $\comp{K}$ is exact if and only if $H^n(\comp{K}) = 0$ for all $n$. 
\end{remark}

\begin{proposition}
    $H^n: \Kom_R \to \catlmod{R}$ is a functor.
\end{proposition}

\begin{proof}
    Define $H^n(f) := f^n$ which is the mapping $H^n(\comp{K}) \to H^n(\comp{L})$. Check according to the definition:
    \begin{itemize}
        \item For any complex $\comp{K}$, $H^n(\comp{K})$ is by definition an $R$-module. 
        \item Identity is mapped to identity, as the same complex has the same cohomology.
        \item $H^n(f \circ g) = H^n(f) \circ H^n(g)$ by the commutativity of the diagram.
    \end{itemize}
\end{proof}

\begin{definition}
    Suppose that $f, g>: \comp{K} \to \comp{L}$ be morphism of complexes. Then $f$ is \textbf{homotopic} to $g$ (denoted $f \sim g$) if there exists $R$-linear maps $\theta^n: K^n \to L^{n-1}$ for all $n \in \Z$, s.t. 
    \[
        f^n - g^n = d_L^{n-1} \circ \theta^n + \theta^{n+1} \circ d_K^n \quad \forall n \in \Z
    \]
    It may be clearer to consult the following diagram:
    \begin{figure}[htbp]
        \centering
        \begin{tikzcd}
            K^{n-1} \arrow[dd, shift left, dashed, "g^{n-1}"] \arrow[rr, "d_K^{n-1}"] & & K^n \arrow[dd, shift left, dashed, "g^{n}"] \arrow[rr, "d_K^{n}"] \arrow[lldd, "\theta^n"] & & K^{n+1} \arrow[dd, shift left, dashed, "g^{n+1}"] \\
            & & & & \\
            L^{n-1} \arrow[uu, shift left, <-, "f^{n-1}"] \arrow[rr, "d_L^{n-1}"] & & L^n \arrow[uu, shift left, <-, "f^{n}"] \arrow[rr, "d_L^{n}"] \arrow[rruu, <-, "\theta^{n+1}"] & & L^{n+1} \arrow[uu, shift left, <-, "f^{n+1}"] \\
        \end{tikzcd}
    \end{figure}
\end{definition}

\begin{remark}
    Homotopy implies that they induces the same cohomology everywhere. That is, if $f \sim g$, Consider for $x \in \ker d_K^n$,
    \[
        f^n(x) - g^n(x) = d_L^{n-1} \circ \theta^n(x) + \theta^{n+1} \circ d_K^n(x) = d_L^{n-1} \circ \theta^n(x) \in \im d_L^{n-1} = \ker d_L^n
    \]
\end{remark}

\begin{remark}
    If $F: \catlmod{R} \to \catlmod{S}$ is an additive functor, then
    \begin{itemize}
        \item If $\comp{K}$ is a complex, then the chain given by $\cdots \overset{F(d^{n-1})}{\too} F(K^n) \overset{F^(d^n)}{\too} F(K^{n+1}) \overset{F(d^{n+1})}{\too} \cdots$ is also a complex. Exactness follows directly from considering the image of the corresponding submodules after applying the functor.
        \item Let $f = (f^n): \comp{K} \to \comp{L}$ be a morphism of complex. Then $F(f)$ is also a morphism of complex. 
        \item If $f \sim g$, then $F(f) \sim F(g)$, as functors by definition commutes with composition; and additive functors commute with addition.
    \end{itemize}
\end{remark}

\begin{proposition}
    Given a short exact sequence of complexes $0 \too \comp{K} \overset{f}{\too} \comp{L} \overset{g}{\too} \comp{M} \too 0$ (tjhat is, for all $n$ the sequence with corresponding maps $0 \too K^n \too L^n \too M^n \too 0$ is exact), then there exists a canonical morphism $H^{n-1}(\comp{M}) \overset{\delta^{n-1}}{\too} H^n(\comp{K})$ for all $n$ s.t. we have a ``long exact sequence in cohomology'':
    \[
        \cdots \too H^n(\comp{K}) \overset{H^n(f)}{\too} H^n(\comp{L}) \overset{H^n(g)}{\too} H^n(\comp{M}) \overset{\delta^n}{\too} H^{n+1}(\comp{K}) \too \cdots
    \]
\end{proposition}

\begin{proof}
    Notice that $(\ker d_M^n/\im d_M^{n-1})$ and $(\ker d_M^{n+1}/\im d_M^n)$ are the kernel and cokernel of $d_M^n$, respectively. Applying Snake Lemma gives the desired result.
\end{proof}

\section{Projective and Injective Resolution$^*$}

\section{Derived Functors$^*$}

\end{document}